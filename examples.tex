\documentclass{article}

\usepackage[hidelinks]{hyperref}
\usepackage[dvipsnames]{xcolor}
\usepackage{amsmath}
\usepackage{amssymb}
\usepackage{amsthm}
\usepackage{bbm}
\usepackage{stmaryrd}
\usepackage{mathtools}
\usepackage{mathpartir}
\usepackage{enumitem}
\usepackage{xspace}
\usepackage{doi}
\usepackage{ifthen}
\usepackage{suffix}

\usepackage[authoryear,sort,square]{natbib}
\bibliographystyle{plainnat}

% Notes
% AKH: Stolen shamelessly from my frequent collaborator, Ethan Cecchetti
\newcommand{\newcommenter}[3]{%
  \newcommand{#1}[1]{%
    \textcolor{#2}{\small\textsf{[#3: {##1}]}}%
  }%
}
\newcommenter{\akh}{purple}{AKH}

%Example Program Macros
\newcommand{\programfont}[1]{\textsf{#1}}


\newcommand{\If}{\programfont{if}\xspace}
\newcommand{\Then}{\programfont{then}\xspace}
\newcommand{\Else}{\programfont{else}\xspace}
\newcommand{\Match}{\programfont{match}\xspace}
\newcommand{\With}{\programfont{with}\xspace}
\newcommand{\In}{\programfont{in}\xspace}
\newcommand{\InQ}{\programfont{in?}\xspace} %for "In Query"
\newcommand{\InP}{\programfont{In}\xspace} %for "In Prop"
\newcommand{\Yes}{\programfont{yes}\xspace}
\newcommand{\No}{\programfont{no}\xspace}
\newcommand{\Left}{\programfont{left}\xspace}
\newcommand{\Right}{\programfont{right}\xspace}
\newcommand{\End}{\programfont{end}\xspace}
\newcommand{\Principal}{\programfont{Principal}\xspace}
\newcommand{\List}{\programfont{List}\xspace}
\newcommand{\Refl}{\programfont{refl}\xspace}
\newcommand{\String}{\programfont{string}\xspace}
\newcommand{\Says}{\programfont{says}\xspace}
\WithSuffix\newcommand\Says*{\ensuremath{\mathrel{\programfont{says}}}}
\newcommand{\Trusts}{\programfont{trusts}\xspace}
\WithSuffix\newcommand\Trusts*{\ensuremath{\mathrel{\programfont{trusts}}}}

% This is a trick I came up with while writing my last paper for writing nice multi-line programs.
% Sadly, it does tend to mess up at least Emacs's automatic spacing, so it goes in its own file.

\newcounter{numlevels}
\newcommand{\newmultilevelprogram}[3][0]{\newcommand{#2}[#1]{\ifthenelse{\value{numlevels} > 0}{\begin{array}[t]{@{}l@{}}}{\begin{array}{l}}\addtocounter{numlevels}{1}#3\addtocounter{numlevels}{-1}\end{array}}}

\newcommand{\newmultilevelprogramwithsuffix}[3][0]{\WithSuffix\newcommand #2 [#1]{\ifthenelse{\value{numlevels} > 0}{\begin{array}[t]{@{}l@{}}}{\begin{array}{l}}\addtocounter{numlevels}{1}#3\addtocounter{numlevels}{-1}\end{array}}}
                                                                                                  

%%% Local Variables:
%%% mode: latex
%%% TeX-master: "examples"
%%% End:


\newcommand{\ITE}[3]{\If~#1~\Then~#2~\Else~#3}
\newmultilevelprogramwithsuffix[3]{\ITE*}{%for "If-Then-Else Multi Line
  \If~#1\\%
  \Then~#2\\%
  \Else~#3%
}
\newcommand{\MatchDP}[5]{\Match~#1 \mathrel{\With} |~\Yes~#2 \mathrel{\Rightarrow} #3 |~\No~#4 \mathrel{\Rightarrow} #5 \End}
\newmultilevelprogramwithsuffix[5]{\MatchDP*}{%for "Match Decision Procedure"
  \Match~#1~\With\\%
  \hspace*{0.5em}|~\Yes~#2 \mathrel{\Rightarrow} #3\\%
  \hspace*{0.5em}|~\No~#4 \mathrel{\Rightarrow} #5\\%
  \End%
}

\newcommand{\MatchSum}[5]{\Match~#1 \mathrel{\With} |~\Left~#2 \mathrel{\Rightarrow} #3 |~\Right~#4 \mathrel{\Rightarrow} #5 \End}
\newmultilevelprogramwithsuffix[5]{\MatchSum*}{%
  \Match~#1~\With\\%
  \hspace*{0.5em}|~\Left~#2 \mathrel{\Rightarrow} #3\\%
  \hspace*{0.5em}|~\Right~#4 \mathrel{\Rightarrow} #5\\%
  \End%
}

\begin{document}

\section{Faceted Value}
\label{sec:faceted-value}

This example is based on faceted information-flow control~\citep{AustinF12}.
In faceted information-flow control, we keep multiple values for a variable, and then show the ``right'' value of the variable to anyone who asks for it.
We have simplified this example somewhat, keeping track of only two values for a faceted value: the ``true'' value and the ``default'' value.
Any principal has access to the value if they are on a whitelist, otherwise they get the default value.
We are also able to provide a simple proof of noninterference for this construct.

$$
\begin{array}{lcl}
  \programfont{facet}_{p,q} &:& p.\mathbb{N} \to p.\mathbb{N} \to \List~\Principal \to q.\mathbb{N}\\
  \programfont{facet}_{p,q}~s~d~\programfont{WL} &=& \addtocounter{numlevels}{1}\ITE*{\InQ~p~\programfont{WL}}{q.s \to p}{q.d \to p}\addtocounter{numlevels}{-1}
\end{array}
$$

$$
\begin{array}{lcl}
  \programfont{facet\_NI}_{p,q} &:& \forall~s_1~s_2~d~\programfont{WL}.~(\InP~p~\programfont{WL}) \lor ((\programfont{facet}_{p,q}~s_1~d~\programfont{WL}) = (\programfont{facet}_{p,q}~s_2~d~\programfont{WL}))\\
  \programfont{facet\_NI}_{p,q} &=& \addtocounter{numlevels}{1}\MatchDP*{\InQ~p~\programfont{WL}}{e}{\Left~e}{n}{\Right~\Refl}\addtocounter{numlevels}{-1}
\end{array}
$$

Note that we have the two principals $p$ and $q$ as subscripts on the programs.
This represents C++-template-style separate compilation, because modality polymorphism is hard in MMT.
\akh{Daniel and Alex, can you expand on that?}

However, we \emph{do} have a list of \Principal, which is a type with decidable equality (which we know because the program \InQ is defined).
This is a reflection of the modalities as a type.
This does a lot of the work for us, but does not allow us to do $\forall p.~p\mathrel{\textsf{says}} \varphi$.
This is clearly important for many applications, but requires a nontrivial extension to MMT.

\section{Examples from the FLAFOL Paper}
\label{sec:examples-from-flafol}

The following examples are from the FLAFOL paper~\citep{HirschACAT20}.
We have dropped the primary example from that paper, since it is inherently about how information-flow labels and authorization interact.
(Though perhaps an Aura-style embedding could let us recover it?~\citep{JiaVMZZSZ08,JiaZ09})

These examples are designed to probe at the edges of noninterference.
How to write noninterference and reason about it is one question we need to answer.

\subsection{Hospital Example}
\label{sec:hospital-example}

This example looks at what happens when trust is ``discoverable;'' that is, when it is just a formula that can be assumed like anything else.
In FLAFOL, we told a story involving a hospital administrator trying to figure out who was paying based on two insurance cards, and a reinsurance scheme (hence the name).
This tells us why this is a plausible thing to want, but the basic structure only involves a disjuntion of trust statements.

$$
\begin{array}{lcl}
  \programfont{hospital}&:& p.(p \Trusts* q_1 \lor p \Trusts* q_2) \to q_1.(q_1 \Trusts* q_2) \to p.(p \Trusts* q_2)\\
  \programfont{hospital}~t_1~t_2 &=&\addtocounter{numlevels}{1}\MatchSum*{t_1}{t_1'}{\programfont{compose}~(q_1.t_2 \xrightarrow{t_1'} p)~t_1'}{t_1'}{t_1'}\addtocounter{numlevels}{-1}
\end{array}
$$

The interesting thing here is what it means about noninterference.
Because we can assume a disjunction of trust, we can use statements of principals that~$p$ does not yet trust to determine~$p$'s beliefs.
(Here, note the use of $t_1'$ in the \Left branch.)

\subsection{Sanitization Example}
\label{sec:sanitization-example}

This example looks at what happens when we allow ``downgrading'' through functions.
In FLAFOL, we told a story about a server which was willing to increase the integrity of user inputs when run through a sanitization function.

$$
\begin{array}{lcl}
  f &:& p.q \Trusts* q \to r \Trusts* p \to p.(q.\String \to \String) \to q.\String \to r.\String\\
  f~\chi_1~\chi_2~\programfont{sanitize}~\programfont{inp} & = & p.\programfont{sanitize}(q \xrightarrow{\chi_1} p.q) \xrightarrow{\chi_2} r
\end{array}
$$

Currently, the sanitization function's type is uninhabited, but not provably so.
However, in the body of $f$, we assume that it is inhabited, and reason from there.
But this allows us to seemingly-break noninterference.

\bibliography{bibliography/akh}
    
\end{document}


%%% Local Variables:
%%% mode: latex
%%% TeX-master: t
%%% End:
