\documentclass[11pt]{article}
\usepackage{mathptmx}



%\usepackage{pslatex}
%\usepackage{fullpage}
\usepackage{xspace}
\usepackage{verbatim}
%\usepackage{wrapfig}
%\usepackage{tgpagella}
\usepackage{url}
%\usepackage[hidelinks=true,bookmarks=true]{hyperref}
\usepackage{graphicx}
\usepackage{subfig}
%\usepackage[compact]{titlesec}
%\usepackage[bf]{caption}
%\usepackage{subcaption}
\usepackage{fancyhdr}
\usepackage[margin=1in]{geometry}
\usepackage{float}
\usepackage[font=small]{caption}

\usepackage{comment}
\usepackage{xcolor}


\colorlet{darkpink}{red!60!purple!50}
\colorlet{lightpurple}{purple!70!blue!100}
\colorlet{teal}{green!70!blue!90}
\colorlet{darkgreen}{green!40!black!100}
\colorlet{orange}{orange!70!red!80}

\newcommand{\pink}[1]{\textcolor{darkpink}{#1}}
\newcommand{\purple}[1]{\textcolor{lightpurple}{#1}}
\newcommand{\teal}[1]{\textcolor{teal}{#1}}
\newcommand{\green}[1]{\textcolor{green}{#1}}
\newcommand{\orange}[1]{\textcolor{orange}{#1}}

\newcommand{\red}[1]{\textcolor{red}{#1}}
\newcommand{\blue}[1]{\textcolor{blue}{#1}}

\newcommand{\ignore}[1]{}

\usepackage{listings,color}


% Ariel (Helvetica)
%\renewcommand{\rmdefault}{phv}
%\renewcommand{\sfdefault}{phv}

% Palatino
\renewcommand{\rmdefault}{ppl}
\renewcommand{\sfdefault}{ppl}

\renewcommand{\textfraction}{0.01}
\setcounter{topnumber}{100}
\setcounter{dbltopnumber}{100}
\setcounter{totalnumber}{100}
\renewcommand{\topfraction}{.9}
\renewcommand{\floatpagefraction}{.9}
\renewcommand{\dbltopfraction}{.9}
\renewcommand{\dblfloatpagefraction}{.9}

\newcounter{cnt}

\setlength{\abovecaptionskip}{0in}
\setlength{\belowcaptionskip}{0in}


% \newcommand{\todo}[1]{}
% \newcommand{\ar}[1]{}
% \newcommand{\done}[1]{}
% \newcommand{\ca}[1]{}
% \newcommand{\jw}[1]{}


%make latex layout figures better
%\renewcommand{\topfraction}{0.85}
%\renewcommand{\textfraction}{0.09}
%\renewcommand{\floatpagefraction}{0.85}


\date{}

\newsavebox{\mylistingbox}

\pagestyle{fancy}
\fancyhead{}
\fancyhf{}
\renewcommand{\headrulewidth}{0pt}
\setcounter{page}{1}
%\usepackage{epstopdf}



\usepackage{paralist}

\begin{document}

\begin{center}
{\LARGE
\textsc{Facilities, Equipment, and Other Resources}
}
\end{center}
\hrule

\vspace{3mm}

The research activities will be conducted in the Computer Science and
Engineering Department at the University at Buffalo (UB), the State
University of New York. The computational needs of the proposed
research and educational activities are within the capabilities of the
computing facilities in the department and university.

\subsection*{Department of Computer Science and Engineering Facilities}

Computer Science and Engineering (CSE) department maintains multiple
information technology services and facilities to support its research
mission. These resources and facilities include (but are not limited to): 

\medskip 
\begin{compactitem}
  \item Storage infrastructures
  \item Compute services
  \item Lab and conference facilities
  \item Desktop infrastructures
  \item Application and database hosting
  \item Network and firewalling
  \item Disaster recovery
  \item Asset and license management / procurement
  \item Print and digital imaging services
  \item Security systems and environmental monitoring 
\end{compactitem}

\medskip \noindent
CSE faculty compute systems include a number of Sun servers, a farm of Dell
servers, and a cluster of Dell nodes. CSE student compute systems include a
number of Sun and Dell servers.
 
CSE research groups occupy over 6,600 square feet of research lab space
ranging from secure, monitored, temperature-controlled data centers to
specialized experimental facilities. CSE instructional labs occupy over
4,000 square feet, configured to serve the characteristic needs of the
courses they host. In addition, CSE's four state-of-the-art data centers
occupy over 2,100 square feet, all of which are environmentally conditioned
and monitored 24/7. 

More than 270 Windows, MacOS and Linux PCs and thin client terminals are
available across the multiple research, instructional and student labs. Each
lab is equipped with printing/imaging and presentation equipment. Internet
connectivity to all lab spaces is provided by 1 Gb/s Ethernet network
connections.

CSE faculty, researchers and students also have access to compute labs
administered by School of Engineering Node Services (SENS) and Computing and
Information Technology (CIT).

Our research lab also maintains three powerful workstations
specifically for running secure multi-party computation experiments
with three computational nodes. The machines are 8-core 3.6GHz
computers with Intel i7-7700 processors and 64GB of RAM running
CentOS and connected through a 1Gb/s LAN.

\subsection*{Center for Computational Research (CCR) Facilities}

The Center for Computational Research (CCR) is a leading academic
supercomputing facility that maintains a high-performance computing
environment, high-end visualization laboratories, and support staff with
expertise in computing, visualization, and networking to assist with
computational needs of faculty and students. The center's extensive
computing facilities, which are housed in a state-of-the-art 4,000 square
feet machine room, include a Linux cluster with more than 8,000 processor
cores. Its peak performance compute capacity exceeds 70 Tflops. The center
also maintains several high-performance storage systems of hundreds of TB,
features a tiled display wall for projecting images in high resolution and a
VisDuo passive stereo system for displaying complex 3D imagery. 

\subsection*{University Information Technologies Facilities}

In addition to the facilities described above, students and faculty have
access to facilities and equipment owned and operated by UB Information
Technology (UBIT). Apart from UBIT running the CCR to support the
university's research mission, it provides many other resources to the
university community. Some of the resources offered by UBIT include:

\medskip 
\begin{compactitem}
  \item Classroom and computer labs equipment;
  \item Voice, data, and video services such as phone, networking, video
    conferencing, wireless, etc.; 
  \item Training through classes, workshops, and documentation;
  \item Shared file space for use by students and faculty.
\end{compactitem}

\end{document}
