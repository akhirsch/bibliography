\input{header}


\begin{document}

\begin{center}
{\LARGE \textsc{Budget Justification}}
\end{center}
\hrule
\vspace{3mm}


\subsection*{Salaries and Wages Graduate PI}
The budget also
allocates summer salary for the PI in the amount of .75 in year 1, 1.5 in year 2, and 1.5 in year 3.
 Note that PI Ziarek plans to
consistently contribute to the project in all of its years, but the
amount of his summer salary is reduced in the beginning of the project to
comply with the NSF's rule of two months per year of NSF funding across all
NSF-sponsored projects.
For the purpose of calculating the NSF two-month limit on salary for senior personnel, the University at Buffalo defines a ``year'' as its fiscal year, which is the consecutive 12-month period beginning July 1 and ending June 30.


\subsection*{Salaries and Wages Graduate Students}
Salaries and wages are based on the 2021 university budget with a 2\%
annual increment.
The budget allocates salary funds for two graduate
research assistants for all 3 years of the project. 
Per University policy, full-time graduate students may work a maximum of 20 hours per week during the academic year and up to 40 hours per week during academic breaks



\subsection*{Fringe Benefits}

Fringe benefit rates are based on the applicable federally negotiated rates
published at\\ 
http://www.buffalo.edu/research/research-services/ub-rates-and-facts/ub-and-rf-rates.html.


\subsection*{Publication Costs}
The publication costs, budgeted at
\$500 per year, are intended to cover expenses associated with journal
publishing, over-the-page-limit conference charges, and possible
expenses associated with poster printing to present research results of the
project.



\subsection*{Supplies and Publication Costs}
The budget allocates \$2,000 in supplies in year 1 of the project which will
be used to purchase 1 laptop for one of the students, as well
as any necessary supplies such as books. 

\subsection*{Travel}
The budget allocates travel funds in each year of the project. 
The budget includes \$5,000 dollars per year for domestic travel.  This includes \$1,000 to travel between CMU and the University at Buffalo for the PIs and
their graduate students to collaborate in person and hold annual retreats.  The remain \$4,000 is budgeted for two people, either both students or the PI and
one student, to attend one domestic conference per year.   The travel budget also includes \$8,000 for international travel for two people, either both students or the PI and
one student, to attend one international conference per year.  Many of the specialized type systems
conferences for session types (e.g. CONCUR) as well as the top programming languages either occur in Europe annually or rotate between the US and Europe.  
 Because the location of conferences is not known more than
a year in advance, the countries that we will travel to are currently unknown.
We also would like to travel to the Grace Hopper
Conference for recruitment of students from under-represented groups - this would be from the domestic travel budget and most likely occur in the first year of the grant..
Because international travel may not yet resume in full capacity in 2022, the travel budget is reduced in year 1 of the project and only encompasses domestic travel.


\subsection*{Tuition}
Tuition is charged onto the grant as per the university's tuition policy.
Tuition for the graduate research assistants is budgeted according to the
university 
rates available at\\
http://www.buffalo.edu/research/research-services/ub-rates-and-facts/
under the tuition-rates-for-budgeting heading.
Full-time out-of-state tuition of 9 credit
hours per academic year semester is budgeted for the graduate research assistants.

\subsection*{Computer Services Fee}
The budget requests funding for the University at Buffalo computer services
required to complete the project. These charges cover the cost of operating
the CSE department's computing equipment, including servers, storage devices,
workstations and printers used for research. The amount assessed in each
year takes into account the number of users of these facilities at a rate of
\$156 per person month of time on the project of CSE department personnel.
The computer services policy and formula are reviewed annually and charged
uniformly to all sponsors. 

\subsection*{Facilities and Administrative Costs}

The University at Buffalo uses a federally negotiated indirect rate of
59.5\% MTDC for all on-campus sponsored research. This rate applies to all
direct costs excluding equipment, tuition, and any subcontract costs in
excess to \$25,000. This indirect rate was most recently negotiated on March 25, 2021.  

\end{document}

