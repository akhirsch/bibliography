\documentclass[11pt]{article}
\usepackage{mathptmx}



%\usepackage{pslatex}
%\usepackage{fullpage}
\usepackage{xspace}
\usepackage{verbatim}
%\usepackage{wrapfig}
%\usepackage{tgpagella}
\usepackage{url}
%\usepackage[hidelinks=true,bookmarks=true]{hyperref}
\usepackage{graphicx}
\usepackage{subfig}
%\usepackage[compact]{titlesec}
%\usepackage[bf]{caption}
%\usepackage{subcaption}
\usepackage{fancyhdr}
\usepackage[margin=1in]{geometry}
\usepackage{float}
\usepackage[font=small]{caption}

\usepackage{comment}
\usepackage{xcolor}


\colorlet{darkpink}{red!60!purple!50}
\colorlet{lightpurple}{purple!70!blue!100}
\colorlet{teal}{green!70!blue!90}
\colorlet{darkgreen}{green!40!black!100}
\colorlet{orange}{orange!70!red!80}

\newcommand{\pink}[1]{\textcolor{darkpink}{#1}}
\newcommand{\purple}[1]{\textcolor{lightpurple}{#1}}
\newcommand{\teal}[1]{\textcolor{teal}{#1}}
\newcommand{\green}[1]{\textcolor{green}{#1}}
\newcommand{\orange}[1]{\textcolor{orange}{#1}}

\newcommand{\red}[1]{\textcolor{red}{#1}}
\newcommand{\blue}[1]{\textcolor{blue}{#1}}

\newcommand{\ignore}[1]{}

\usepackage{listings,color}


% Ariel (Helvetica)
%\renewcommand{\rmdefault}{phv}
%\renewcommand{\sfdefault}{phv}

% Palatino
\renewcommand{\rmdefault}{ppl}
\renewcommand{\sfdefault}{ppl}

\renewcommand{\textfraction}{0.01}
\setcounter{topnumber}{100}
\setcounter{dbltopnumber}{100}
\setcounter{totalnumber}{100}
\renewcommand{\topfraction}{.9}
\renewcommand{\floatpagefraction}{.9}
\renewcommand{\dbltopfraction}{.9}
\renewcommand{\dblfloatpagefraction}{.9}

\newcounter{cnt}

\setlength{\abovecaptionskip}{0in}
\setlength{\belowcaptionskip}{0in}


% \newcommand{\todo}[1]{}
% \newcommand{\ar}[1]{}
% \newcommand{\done}[1]{}
% \newcommand{\ca}[1]{}
% \newcommand{\jw}[1]{}


%make latex layout figures better
%\renewcommand{\topfraction}{0.85}
%\renewcommand{\textfraction}{0.09}
%\renewcommand{\floatpagefraction}{0.85}


\date{}

\newsavebox{\mylistingbox}

\pagestyle{fancy}
\fancyhead{}
\fancyhf{}
\renewcommand{\headrulewidth}{0pt}
\setcounter{page}{1}
%\usepackage{epstopdf}




\begin{document}

\begin{center}
{\LARGE \textsc{Data Management Plan}}
\end{center}
\hrule

\vspace{3mm}

\subsection*{Types of Data}  
The data generated and produced under this grant will consist of the typing rules for the type theory, the formal semantics for the process calculus,
an implementation in Rust, benchmark programs written in Rust to evaluate the expressivity of the type system, and a set of mechanized proofs in Coq. 

\subsection*{Data Format and Content}
 There is no set standard for the types of data expected to be generated under this proposal. However, the PIs will use
best practices for formatting and presenting the data generated and collected. The code, files containing the
formalism,
and data generated will be made available on the PIs' personal websites as well as the
project git repository.  The PIs' websites are hosted by Carnegie Mellon University and University at Buffalo respectively and the project
git repository is hosted on GitHub.

\subsection*{Facilities for Storage and Preservation}
Course materials will be available mostly via Carnegie Mellon University's and University at Buffalo's course management systems and
websites as well as Piazza. Other  products of the proposed research program will be stored on a
departmental servers, where everything is regularly backed-up to a long-term
archive. Archives will be kept permanently at University at Buffalo by the CSE IT staff and at Carnegie Mellon University.

\subsection*{Policies of Access} 
The proposed research program does not involve collecting sensitive data for evaluation; the proposed mechanisms are agnostic to the content of the data, {\em i.e.}, there is no need to look into the content of the data and analyze it.  Thus, all data gathered will be made publicly available.  The PIs plan to make both the raw data as well as processed data available for other researchers. Scripts for processing the raw data will be made available as well. 



\subsection*{Policies for Reuse}  
Because the proposed research program does not involve collecting sensitive
data for evaluation, data will be made available for reuse by other researchers.

\subsection*{Plans for Archiving} 
Data will be archived on the project website as well as with digital
libraries that support data submission along with publications.  The PIs currently store data gathered in
version controlled repositories. These repositories are synced with
publication repositories.  This allows for easy correlation of which data was used for which publication.  In addition, data is organized
by publication on the project website.  All data gathered, even if it is not
used in a publication, is still associated with that publication and made available on the project website.  

\subsection*{Management Roles}
The PIs will be the primary managers of the data gathered and stored throughout the project.  After the grant period ends, the data collected will continue
to be made available through the websites it was originally hosted on. 

\end{document}

