\subsection{Kinding System}
\label{sec:kind-system}
 We have four kinds of types in our language:
\begin{itemize}
  \item Locations $*_\ell$
  \item Sets of locations $*_s$
  \item Choreographic types $*$
  \item Local program types $*_e$
\end{itemize}
As with our language syntax, much of the syntax of our types are reminiscent of Pirouette and earlier typed choreographic languages, as well as System F.

\paragraph{Locations and Location Sets}
Location names are represented by the kind $*_\ell$.
This kind of type consists of precisely the concrete locations $L \in \Locations$, and the location variables $\ell \knd *_\ell$ that have been bound by the for-all quantifier.

Sets of location names belong to the kind $*_s$.
In particular, this kind consists of finite sets of elements of the $*_\ell$ kind, as introduced by the empty set $\varnothing$, singleton set $\{\ell\}$, and set union $\rho_1 \cup \rho_2$ constructors.

\paragraph{Choreographic Types}
There are three choreographic types of kind $*$ used to describe \langname programs.
The located type $t_e @ \ell$ represents a local program of type $t_e$ located at $\ell$.
Both $t_e$ and $\ell$, along with all other types, may either be concrete, or contain bound type variables.
The choreographic function type $\tau_1 \to \tau_2$ represents (possibly higher-order) choreographies which take a value of type $\tau_1$ as input, and result in a value of type $\tau_2$.
The universal quantifier type $\forall \alpha \knd \kappa.\tau$ is used to represent choreographic programs which implement parametric polymorphism.
Specifically, a program which take as input a type $t$ of kind $\kappa$, and results in a value of type $\subst{\tau}{\alpha}{t}$.

These three types have the following kinding rules:
\begin{mathpar}
  \infer[TyAt]{\localkinded{\proj{\Gamma}{}}{t_e} \\
    \chorkinded{\Gamma}{\ell}{*_\ell}}
    {\chorkinded{\Gamma}{t_e @ \ell}{*}} \and

  \infer[TyArrow]{\chorkinded{\Gamma}{\tau_1}{*} \\
    \chorkinded{\Gamma}{\tau_2}{*}}
    {\chorkinded{\Gamma}{\tau_1 \to \tau_2}{*}} \and

  \infer[TyAbs]{\chorkinded{\Gamma, \alpha \knd \kappa}{\tau}{*}}
    {\chorkinded{\Gamma}{\forall \alpha \knd \kappa.\tau}{*}} \and 
\end{mathpar}
For example, the type of a choreographic program which sends a value of type $\eta$ from $\alpha$ to $\beta$, for any possible types $\eta$ and locations $\alpha$ and $\beta$, can be written as follows:
\begin{mathpar}
  \forall \eta \knd *_e.\forall \alpha \knd *_\ell.\forall \beta \knd *_\ell.\eta @ \alpha \to \eta @ \beta
\end{mathpar}

\paragraph{Local Types}
Kinds of type $*_e$ are precisely the types included in the local language.
The local language may or may-not include the capability of binding type variables, and this decision has no explicit effect on the choreographic types.
However, local types included in our type system may use type variables which have been bound by the choreographic for-all type.
For instance, recall Example \ref{ex:st-lambda}, a simply-typed lambda calculus with no polymorphism.
Although type variables are not bound by any type constructors of this local language, type variables can still be explicitly used as in the following type representing a polymorphic function application at a concrete location $L$:
\begin{mathpar}
  \forall \eta_1 \knd *_e.\forall \eta_2 \knd *_e.(\eta_1 \to \eta_2) @ L \to \eta_1 @ L \to \eta_2 @ L
\end{mathpar}

\subsection{Type System}
\label{sec:type-system}

\todo How polymorphism is handled

\todo How \& why the tell-let typing rules work
\ethan{Might want to move this higher up since it's our major innovation.}

\todo Statement of important metatheory theorems

%%% Local Variables:
%%% mode: latex
%%% TeX-master: "conference"
%%% End:
