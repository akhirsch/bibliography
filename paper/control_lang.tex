\subsection{Control Language}
\label{sec:control-lang}

\todo purpose and design of control language

\todo explanation of AmI and reason it's needed

\begin{figure}
  \begin{syntax}
    \category[Control Expression]{E}
    \alternative{X} \alternative{\CtrlFail} \alternative{\Ret{e}} \alternative{\CtrlSeq{E_1}{E_2}} \alternative{\CtrlFun{X}{E}} \alternative{\CtrlRec{X}{E}} \alternative{E_1~E_2} \\
    \alternative{\SendTo{E}{\ell}} \alternative{\RecvFrom{\ell}} \\
    \alternative{\ChooseFor{d}{\ell}{E}} \alternative{\AllowChoice{\ell}{?E_1}{?E_2}} \\
    \alternative{\CtrlITE{E}{E_1}{E_2}} \\
    \alternative{\CtrlTFun{\alpha}{E}} \alternative{E~t} \\
    \alternative{\CtrlLetIn{\Ret{x}}{E_1}{E_2}}\\
    \alternative{\LetSendIn{\alpha \knd *_\ell}{E_1}{\rho}{E_2}} \alternative{\LetRecvIn{\alpha \knd *_\ell}{\ell}{E}}\\
    \alternative{\LetSendIn{\alpha \knd *_e}{E_1}{\rho}{E_2}} \alternative{\LetRecvIn{\alpha \knd *_e}{\ell}{E}}\\
    \alternative{\AmI{\ell}{E_1}{E_2}}

    \category[Possible Control Expression]{?E}
    \alternative{\CtrlNone}
    \alternative{E}

    \category[Systems]{\Pi}
    \alternative{L_1 \triangleright E_1 \;\|\; \ldots \;\|\; L_n \triangleright E_n}
  \end{syntax}

  \caption{Control Expressions}
  \label{fig:control-lang-syntax}
\end{figure}

\paragraph{Operational Semantics}
\label{sec:control-lang-semantics}


\begin{syntax}
\category[Label]{l}
\alternative{\iota} \alternative{V \sendsto L} \alternative{L;V \sendsto} \alternative{d \sendsto L} \alternative{L;d \sendsto} \alternative{t \sendsto \rho} \alternative{L;t \sendsto}
\end{syntax}

\begin{mathparpagebreakable}
    \infer[RetE]{e_1 \localstep e_2}
    {\Ret{e_1} \ctrlstep[\iota] \Ret{e_2}} \and

    \infer[Seq]{E_1 \ctrlstep[l] E_1'}
    {\CtrlSeq{E_1}{E_2} \ctrlstep[l] \CtrlSeq{E_1'}{E_2}} \and

    \infer[SeqV]{\text{Value}(V)}
    {\CtrlSeq{V}{E_2} \ctrlstep[\iota] E_2} \and

    \infer[AppFun]{E_1 \ctrlstep[l] E_1'}
    {E_1~E_2 \ctrlstep[l] E_1'~E_2} \and

    \infer[AppArg]{\text{Value}(V) \\
      E_2 \ctrlstep[l] E_2'}
    {V~E_2 \ctrlstep[l] V~E_2'} \and

    \infer[App]{\text{Value}(V)}
    {(\CtrlFun{X}{E})~V \ctrlstep[\iota] \subst{E}{X}{V}} \and

    \infer[Rec]{~}
    {\mu X.E \ctrlstep[\iota] \subst{E}{X}{\mu X.E}} \and
   
    \infer[Send]{E \ctrlstep[l] E'}
    {\SendTo{E}{L'} \ctrlstep[l] \SendTo{E'}{L'}} \and

    \infer[SendV]{\text{Value}(V)}
    {\SendTo{V}{L'} \ctrlstep[V \sendsto L] \CtrlFail} \and

    \infer[Recv]{\text{Value}(V)}
    {\RecvFrom{L'} \ctrlstep[L';V \sendsto] V} \and

    \infer[Choose]{~}
    {\ChooseFor{d}{L'}{E} \ctrlstep[d \sendsto L'] E} \and

    \infer[AllowChoiceL]{~}
    {\AllowChoice{L'}{E_1}{E_2} \ctrlstep[L';\Left \sendsto] E_1} \and

    \infer[AllowChoiceR]{~}
    {\AllowChoice{L'}{E_1}{E_2} \ctrlstep[L';\Right \sendsto] E_2} \and

    \infer[AppTyFun]{E \ctrlstep[l] E'}
    {E~t \ctrlstep[l] E'~t} \and

    \infer[AppTy]{~}
    {(\CtrlFun{\alpha}{E})~t \ctrlstep[\iota] \subst{E}{\alpha}{t}} \and

    \infer[LetRet]{E_1 \ctrlstep[l] E_1'}
    {\CtrlLetIn{\Ret{x}}{E_1}{E_2} \ctrlstep[l] \CtrlLetIn{\Ret{x}}{E_1'}{E_2}} \and

    \infer[LetRetV]{\text{Value}(v)}
    {\CtrlLetIn{\Ret{x}}{\Ret{v}}{E} \ctrlstep[\iota] \subst{E}{x}{v}} \and

    \infer[SendTy]{E_1 \ctrlstep[l] E_1'}
    {\LetSendIn{\alpha \knd \kappa}{E_1}{\rho}{E_2} \ctrlstep[l] \LetSendIn{\alpha \knd \kappa}{E_1'}{\rho}{E_2}} \and

    \infer[SendTyV]{~}
    {\LetSendIn{\alpha \knd \kappa}{\Ret{\say{t}}}{\rho}{E} \ctrlstep[t \sendsto \rho] \subst{E}{\alpha}{t}} \and

    \infer[RecvTy]{~}
    {\LetRecvIn{\alpha \knd \kappa}{L}{E} \ctrlstep[L';t \sendsto] \subst{E}{\alpha}{t}} \and

    \infer[AmIL]{~}
    {\AmI{L}{E_1}{E_2} \ctrlstep[\iota] E_1} \and

    \infer[AmIR]{L \neq L'}
    {\AmI{L}{E_1}{E_2} \ctrlstep[\iota] E_2} \and
\end{mathparpagebreakable}


%%% Local Variables:
%%% mode: latex
%%% TeX-master: "conference"
%%% End:
