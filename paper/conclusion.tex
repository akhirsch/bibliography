\section{Conclusion}
\label{sec:conclusion}

\akh{I'm moving related work to its own section.
	The conclusion should be where we discuss the insights of this work.
	Remember, the technical component isn't the thing people will remember, it's the insights.
	Tell them what they are, and remind them why the technical work we did validated them.}

\todo Related works

\todo Why we do what they can't

\todo Why these abilities are useful

\todo Future work
\begin{itemize}
	\item Data types at the choreographic level: Products, sum, recursive
	\item Prevent bad usage of tell-let variables within the type system. Increases compositionality
	\item Mutable references? (Symphony)
	\item Explicit delegation capabilities? (Alice \& Bob)
\end{itemize}

The key components of this choreography which are not expressible in previous works are the \textsf{acquireWorker} function, which allows the pool manager to \emph{dynamically} select a worker location, and the ability to communicate location names between nodes, enabling the chosen worker name to be shared with the client.
Previous works, notably PolyChor$\lambda$, allow for \emph{delegation} of tasks to workers as in our thread pool.
However in this language, as locations are part of the type system, types may not depend on values (that is, it is not a dependent type system), and there is no mechanism to reflect values into types, the locations must always be statically known, precluding the ability to \emph{dynamically} select location names at runtime.
(Focus on what we can do rather than why they can't, move to related works possibly)

%%% Local Variables:
%%% mode: latex
%%% TeX-master: "conference"
%%% End: