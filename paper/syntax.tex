\subsection{Syntax}
\label{sec:syntax}
The full syntax of \langname is presented in \todo.
As we extend the Pirouette language, much of our syntax is identical or similar.
As in Pirouette, we include both choreographic program variables written in uppercase Latin characters ($X,Y,F,\ldots$) and local program variables written in lowercase Latin characters ($x,y,f,\ldots$).
Both choreographic and local type variables, including location variables, are denoted with lowercase Greek characters ($\alpha,\beta,\ldots$).

\todo Explain base choreographic stuff: $\ell.e$, $C_1 \ChorSend[\ell_1] \ell_2$, $\LetIn{\ell.x}{C_1}{C_2}$, $\syncs{\ell_1}{d}{\ell_2}$, $\lambda X \ty \tau.C$, $\mu X \ty \tau.C$

\todo Explain location and type abstraction $\Lambda \alpha \knd \kappa.C$

\todo Explain tell-let binding $\LetTellIn{\alpha \knd *_\ell}{\ell}{C_1}{\rho}{C_2}$

\paragraph{Location Substitution}
\ethan{I cut this discussion down and cleaned it up a bit.
  It also feels like the wrong place, but it needs to be somewhere. Maybe in the semantics section?}
While substitution and variable renaming can largely be handled in the standard way,
care must be taken when substituting \emph{locations} to avoid variable capture that is not otherwise possible.
Consider the following choreography with a location variable~$\alpha$ and concrete location~$L$.
$$\begin{array}{l}
  C = \begin{array}[t]{@{}l@{}}
    \LetN~\begin{array}[t]{@{}l@{}}
      \alpha.x\ty\Int \ChorDef \alpha.2 \\
      L.x\ty\Int \ChorDef L.3 \\
      \alpha.y \ChorDef L.x \ChorSend[L] \alpha \\
    \end{array} \\
    \In~\alpha.(x + y)
  \end{array}
\end{array}$$
Note that because $\alpha$ and $L$ are distinct, the local variables $\alpha.x$ and $L.x$ are also distinct,
so this choreography should always produce $\alpha.5$ regardless of the concrete location~$\alpha$.

Now suppose~$\alpha$ resolves to~$L$.
How do we define $\subst{C}{\alpha}{L}$?
One might naively think that, since~$\alpha$ does not clash with any other variables, we can simply replace all instances of~$\alpha$ with~$L$.
However, doing so changes the meaning of the program as the previously-distinct local variables $\alpha.x$ and $L.x$ collapse.
The variable $\alpha.x$ is incorrectly captured by the definition of $L.x$ and the choreography would wrongly evaluate to~$L.6$.

Note that, when substituting $L$ in for $\alpha$, capture can occur when binding either $\alpha.x$ or $L.x$.
When substituting a binding of $\alpha.x$, any free instances of $L.x$ in the body will be captured,
and so too will free instances of $\alpha.x$ be captured when substituting a binding of $L.x$.

To avoid this namespace capture, substitutions of locations in local let expressions must rename variables when binding within either namespace.
Notationally, we write $\hsubst{C}{\ell}{x}{e}$ to denote standard capture-avoiding substitution of the local variable~$x$ at the location (or location variable)~$\ell$.
Safe location substitution for local let bindings is then defined follows.
\begin{align*}
  \subst{(\LetIn{\alpha.x\ty t}{C_1}{C_2})}{\alpha}{\ell} & = \LetIn*{\ell.x\ty t}{\subst{C_1}{\alpha}{\ell}}{\subst{C_2}{\alpha}{\ell}}
  && \text{if}~\ell.x \notin \fv(C_2)
  \\[1ex]
  \subst{(\LetIn{\alpha.x\ty t}{C_1}{C_2})}{\alpha}{\ell}
  & = \LetIn*{\ell.y\ty t}{\subst{C_1}{\alpha}{\ell}}{\subst{\hsubst{C_2}{\alpha}{x}{y}}{\alpha}{\ell}}
  && \begin{array}{@{}l@{}}
    \text{if}~\ell.x \in \fv(C_2), \\
    \text{and}~\alpha.y,\ell.y \notin \fv(C_2)
  \end{array}
  \\[1ex]
  \subst{(\LetIn{\ell.x\ty t}{C_1}{C_2})}{\alpha}{\ell}
  & = \LetIn*{\ell.y\ty t}{\subst{C_1}{\alpha}{\ell}}{\subst{\hsubst{C_2}{\ell}{x}{y}}{\alpha}{\ell}}
  && \begin{array}{@{}l@{}}
    \text{if}~\alpha.x \in \fv(C_2), \\
    \text{and}~\alpha.y,\ell.y \notin \fv(C_2)
  \end{array}
  \\[1ex]
  \subst{(\LetIn{\beta.x\ty t}{C_1}{C_2})}{\alpha}{\ell} & = \LetIn*{\beta.x\ty t}{\subst{C_1}{\alpha}{\ell}}{\subst{C_2}{\alpha}{\ell}}
  && \begin{array}{@{}l@{}}
    \text{if}~\alpha \neq \beta ~\text{and} \\
    \text{either}~ \beta \neq \ell ~\text{or}~ \alpha.x \notin \fv(C_2)
  \end{array}
\end{align*}

\begin{figure}[h]
  \begin{syntax}
    \abstractCategory[Local Expressions]{e}
    \abstractCategory[Local Values]{v}
    \abstractCategory[Local Variables]{x, y, \ldots}
    \\

    \category[Synchronization Labels]{d}
    \alternative{\Left}
    \alternative{\Right}

    \abstractCategory[Choreographic Variables]{X, Y, \ldots}

    \category[Choreographies]{C}
    \alternative{X}
    \alternative{\ell.e}
    \alternative{\Fun{X \ty \tau}{C}}
    \alternative{\Rec{X \ty \tau}{C}}
    \alternative{C_1~C_2}\\
    \alternative{C \ChorSend[\ell_1] \ell_2}
    \alternative{\syncs{\ell_1}{d}{\ell_2} \seq C}\\
    \alternative{\ITE{C}{C_1}{C_2}}\\
    \alternative{\TFun{\alpha \knd \kappa}{C}}
    \alternative{C~t}\\
    \alternative{\LetIn{\ell.x \ty t_e}{C_1}{C_2}}\\
    \alternative{\LetTellIn{\alpha \knd \kappa}{\ell}{C_1}{\rho}{C_2}}\\
%    \alternative{\LetTellIn{\alpha \knd *_e}{\ell}{C_1}{\rho}{C_2}}

    \category[Choreography Values]{V}
    \alternative{L.v}
    \alternative{\Fun{X \ty \tau}{C}}
    \alternative{\TFun{\alpha \knd \kappa}{C}}
  \end{syntax}

  \caption{Syntax of Choreographies}
  \label{fig:abstract-syntax}
\end{figure}

\begin{figure}[h]
  \begin{syntax}
    \category[Kinds]{\kappa}
    \alternative{*}
    \alternative{*_\ell}
    \alternative{*_s}
    \alternative{*_e}
    \\
    
    \abstractCategory[Type Variables]{\alpha, \beta, \ldots}
    
    \abstractCategory[Local Types]{t_e}

    \categoryFromSet[Locations]{L}{\Locations}
    
    \category[Choreography Types]{t, \ell, \rho, \tau}
    \alternative{\alpha}
    \alternative{t_e @ \ell}
    \alternative{\tau_1 \rightarrow \tau_2}
    \alternative{\forall \alpha \knd \kappa.\tau}
    \\
    \alternative{L}
    \alternative{\varnothing}
    \alternative{\{\ell\}}
    \alternative{\rho_1 \cup \rho_2}
    \\
    \category[Choreography Kinding Contexts]{\Gamma}
    \alternative{\cdot}
    \alternative{\Gamma, \alpha \knd \kappa}
    
    \category[Choreography Typing Contexts]{\Delta}
    \alternative{\cdot}
    \alternative{\Delta, X \ty \tau}
    
    \category[Locally-Bound Typing Contexts]{\Sigma}
    \alternative{\cdot}
    \alternative{\Sigma, \ell.x \ty t_e}
    \\
    \category[Local Kinding Contexts]{\Gamma_e}
    \alternative{\cdot}
    \alternative{\Gamma_e, \alpha}
    
    \category[Local Typing Contexts]{\Sigma_e}
    \alternative{\cdot}
    \alternative{\Sigma_e, x \ty t_e}
  \end{syntax}

  \caption{Syntax of Kinds, Types, and Contexts}
  \label{fig:types}
\end{figure}

%%% Local Variables:
%%% mode: latex
%%% TeX-master: "conference"
%%% End:
