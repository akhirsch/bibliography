\subsection{Proof of Theorem \ref{thm:lifting-and-lowering}}
\begin{lem}[$\lessnd$ is Reflexive]
\label{lem:lessnd-refl}
For all control expressions $E$, $E \lessnd E$.
\end{lem}

\begin{lem}[Local Substitution Preserves $\lessnd$]
\label{lem:lessnd-subst}
If $E_1 \lessnd E_2$, then for any local variable $x$ and local expression $e$, $\subst{E_1}{x}{e} \lessnd \subst{E_2}{x}{e}$.
\end{lem}

\begin{lem}[Values are Deterministic]
\label{lem:values-determined}
If $V$ is a control expression value and $V \lessnd E$ or $E \lessnd V$, then $E = V$.
\end{lem}

\begin{lem}[Internal Control Step Lifting]
\label{lem:internal-control-lifting}
If $E_1 \lessnd E_2$ and $E_1 \ctrlstep[\iota] E_1'$, there is some $E_2'$ such that $E_1' \lessnd E_2'$ and $E_2 \ctrlstep[\iota] E_2'$.
\end{lem}
We proceed by induction on the control step $E_1 \ctrlstep[\iota] E_1'$.
\begin{description}
  \item[\textsc{RetE}]
    Let $E_1 = \Ret{e_1}$ and $E_1' = \Ret{e_1'}$, where $e_1 \localstep e_1'$.
    By examination of $\lessnd$ we have $E_2 = \Ret{e_1}$, so $\Ret{e_1'}$ suffices.
  \item[\textsc{Seq}]
    Let $E_1 = \CtrlSeq{E_{1,1}}{E_{2,1}}$ and $E_1' = \CtrlSeq{E_{1,1}'}{E_{2,1}}$ where $E_{1,1} \ctrlstep[\iota] E_{1,1}'$.
    By examination of $\lessnd$ we have $E_2 = \CtrlSeq{E_{1,2}}{E_{2,2}}$ where $E_{1,1} \lessnd E_{1,2}$ and $E_{2,1} \lessnd E_{2,2}$.
    By induction there is some $E_{1,2}'$ such that $E_{1,2}' \lessnd E_{1,1}'$ and $E_{1,2} \ctrlstep[\iota] E_{1,2}'$.
    Therefore $\CtrlSeq{E_{1,2}'}{E_{2,2}}$ suffices.
  \item[\textsc{SeqV}]
    Let $E_1 = \CtrlSeq{V}{E_1}$ and $E_1' = V$.
    By examination of $\lessnd$ and Lemma \ref{lem:values-determined} we have $E_2 = \CtrlSeq{V}{E_2}$ where $E_1 \lessnd E_2$.
    Therefore $V$ suffices.
  \item[\textsc{AppFun}]    
    Let $E_1 = E_{1,1}~E_{2,1}$ and $E_1' = E_{1,1}'~E_{2,1}$ where $E_{1,1} \ctrlstep[\iota] E_{1,1}'$.
    By examination of $\lessnd$ we have $E_2 = E_{1,2}~E_{2,2}$ where $E_{1,1} \lessnd E_{1,2}$ and $E_{2,1} \lessnd E_{2,2}$.
    By induction there is some $E_{1,2}'$ such that $E_{1,2}' \lessnd E_{1,1}'$ and $E_{1,2} \ctrlstep[\iota] E_{1,2}'$.
    Therefore $E_{1,2}'~E_{2,2}$ suffices.
  \item[\textsc{AppArg}]
    Let $E_1 = V~E_1$ and $E_1' = V~E_1'$, where $E_1 \ctrlstep[\iota] E_1'$.
    By examination of $\lessnd$ and and Lemma \ref{lem:values-determined} we have $E_2 = V~E_2$ where $E_1 \lessnd E_2$.
    By induction there is some $E_2'$ such that $E_1' \lessnd E_2'$ and $E_2 \ctrlstep[\iota] E_2'$.
    Therefore $V~E_2'$ suffices.
  \item[\textsc{Send}]
    Let $E_1 = \SendTo{E_1}{L'}$ and $E_1' = \SendTo{E_1'}{L'}$, where $E_1 \ctrlstep[\iota] E_1'$.
    By examination of $\lessnd$ we have $E_2 = \SendTo{E_2}{L'}$, where $E_1 \lessnd E_2$.
    By induction there is some $E_2'$ such that $E_1' \lessnd E_2'$ and $E_2 \ctrlstep[\iota] E_2'$, so $\SendTo{E_2'}{L'}$ suffices.
  \item[\textsc{AppTFun}]    
    Let $E_1 = E_1~t$ and $E_1' = E_1'~E_t$ where $E_1 \ctrlstep[\iota] E_1'$.
    By examination of $\lessnd$ we have $E_2 = E_2~t$ where $E_1 \lessnd E_2$.
    By induction there is some $E_2'$ such that $E_1' \lessnd E_2'$ and $E_2 \ctrlstep[\iota] E_2'$.
    Therefore $E_2'~t$ suffices.
  \item[\textsc{LetRet}]    
    Let $E_1 = \CtrlLetIn{\Ret{x}}{E_{1,1}}{E_{2,1}}$ and $E_1' = \CtrlLetIn{\Ret{x}}{E_{1,1}'}{E_{2,1}}$ where $E_{1,1} \ctrlstep[\iota] E_{1,1}'$.
    By examination of $\lessnd$ we have $E_2 = \CtrlLetIn{\Ret{x}}{E_{2,1}}{E_{2,2}}$ where $E_{1,1} \lessnd E_{1,2}$ and $E_{2,1} \lessnd E_{2,2}$.
    By induction there is some $E_{2,1}'$ such that $E_{1,1}' \lessnd E_{2,1}'$ and $E_{2,1} \ctrlstep[\iota] E_{2,1}'$.
    Therefore $\CtrlLetIn{\Ret{x}}{E_{2,1}'}{E_{2,2}}$ suffices.
  \item[\textsc{LetRetV}]    
    Let $E_1 = \CtrlLetIn{\Ret{x}}{\Ret{v}}{E_1}$ and $E_1' = \subst{E_1}{x}{v}$.
    By examination of $\lessnd$ we have $E_2 = \CtrlLetIn{\Ret{x}}{\Ret{v}}{E_2}$ where $E_1 \lessnd E_2$.
    By Lemma \ref{lem:lessnd-subst} $\subst{E_1}{x}{v} \lessnd \subst{E_2}{x}{v}$, so $\subst{E_2}{x}{v}$ suffices.
  \item[\textsc{SendTy}]
    Let $E_1 = \LetSendIn{\alpha \knd \kappa}{E_{1,1}}{\rho}{E_{2,1}}$ and $E_1' = \LetSendIn{\alpha \knd \kappa}{E_{1,1}'}{\rho}{E_{2,1}}$, where $E_{1,1} \ctrlstep[\iota] E_{1,1}'$.
    By examination of $\lessnd$ we have $E_2 = \LetSendIn{\alpha \knd \kappa}{E_{1,2}}{\rho}{E_{2,2}}$ where $E_{1,1} \lessnd E_{2,1}$ and $E_{2,1} \lessnd E_{2,2}$.
    By induction there is some $E_{1,2}'$ such that $E_{1,1}' \lessnd E_{1,2}'$ and $E_{1,2} \ctrlstep[\iota] E_{1,2}'$.
    Therefore $\LetSendIn{\alpha \knd \kappa}{E_{1,2}'}{\rho}{E_{2,2}}$ suffices.
  \item[\textsc{AmIL}]
    Let $E_1 = \AmI{L}{E_{1,1}}{E_{2,1}}$ and $E_1' = E_{1,1}$.
    By examination of $\lessnd$ we have $E_2 = \AmI{L}{E_{1,2}}{E_{2,2}}$ where $E_{1,1} \lessnd E_{1,2}$ and $E_{2,1} \lessnd E_{2,2}$.
    Therefore $E_{1,2}$ suffices.
  \item[\textsc{AmIR}]
  Let $E_1 = \AmI{L'}{E_{1,1}}{E_{2,1}}$ and $E_1' = E_{2,1}$.
  By examination of $\lessnd$ we have $E_2 = \AmI{L'}{E_{1,2}}{E_{2,2}}$ where $E_{1,1} \lessnd E_{1,2}$ and $E_{2,1} \lessnd E_{2,2}$.
  Therefore $E_{2,2}$ suffices.
\end{description}

\begin{lem}[Synchronized Internal Control Step Lifting]
\label{lem:internal-sync-control-lifting}
If $E_1 \lessnd E_2$ and $E_1 \ctrlstep[\iotasync] E_1'$, there is some $E_2'$ such that $E_1' \lessnd E_2'$ and $E_2 \ctrlstep[\iotasync] E_2'$.
\end{lem}
We proceed by induction on the control step $E_1 \ctrlstep[\iotasync] E_1'$.
\begin{description}
  \item[\textsc{Seq}, \textsc{AppFun}, \textsc{Send}, \textsc{AppTFun}, \textsc{LetRet}, \textsc{SendTy}]
    Analogous to the corresponding cases in Lemma \ref{lem:internal-control-lifting}, replacing $\iota$ with $\iotasync$.
  \item[\textsc{App}]
    Let $E_1 = (\CtrlFun{X}{E})~V$ and $E_1' = \subst{E}{X}{V}$.
    By examination of $\lessnd$ and Lemma \ref{lem:values-determined} we have $E_2 = (\CtrlFun{X}{E})~V$, and so $\subst{E}{X}{V}$ suffices.
  \item[\textsc{Rec}]
    Let $E_1 = \CtrlRec{X}{E}$ and $E_1' = \subst{E}{X}{\CtrlRec{X}{E}}$.
    By examination of $\lessnd$ we have $E_2 = \CtrlRec{X}{E}$, and so $\subst{E}{X}{\CtrlRec{X}{E}}$ suffices.
  \item[\textsc{AppT}]
    Let $E_1 = (\CtrlTFun{\alpha}{E})~t$ and $E_1' = \subst{E}{\alpha}{t}$.
    By examination of $\lessnd$ we have $E_2 = (\CtrlTFun{\alpha}{E})~t$, and so $\subst{E}{\alpha}{t}$ suffices.
\end{description}

\begin{lem}[Send Control Step Lifting]
\label{lem:send-control-lifting}
If
\begin{itemize}
  \item $E_1 \ctrlsteploc[v \sendsto L_2]{L_1} E_1'$,
  \item $E_1 \lessnd E_{1 \leq}$,
  \item $E_2 \ctrlsteploc[L_1;v \sendsto]{L_2} E_2'$,
  \item $E_2 \lessnd E_{2 \leq}$, and
  \item $L_1 \neq L_2$,
\end{itemize}
then there is some $E_{1 \leq}'$ and $E_{2 \leq}'$ such that
\begin{itemize}
  \item $E_{1 \leq} \ctrlsteploc[v \sendsto L_2]{L_1} E_{1 \leq}'$,
  \item $E_1' \lessnd E_{1 \leq}'$,
  \item $E_{2 \leq} \ctrlsteploc[L_1;v \sendsto]{L_2} E_{2 \leq}'$, and
  \item $E_2' \lessnd E_{2 \leq}'$.
\end{itemize}
\end{lem}
We proceed by induction on both control steps.
\begin{description}
  \item[\textsc{Seq}, \textsc{AppFun}, \textsc{Send}, \textsc{AppTFun}, \textsc{LetRet}, \textsc{SendTy}]
    When either of the steps have one of the preceding cases, the proof is analogous to the corresponding case in Lemma \ref{lem:internal-control-lifting}.
  \item[\textsc{SendV} and \textsc{Recv}]
    Consider the case when $E_1 = \SendTo{\Ret{v}}{L2}$ and $E_2 = \RecvFrom{L1}$.
    By examination of $\lessnd$, $E_1$ and $E_2$ are only related to themselves, so $E_{1 \leq}' = \CtrlFail$ and $E_{2 \leq}' = \Ret{v}$ are satisfactory.
\end{description}

\todo
\ashley{I want to restate the theorem here, and I need a clean solution to keep the numbering correct}
\label{sec:lifting-proof}
\paragraph{Proof of \textsc{Lifting}}
Suppose that $\Pi_1 \lessnd \Pi_2$ and $\Pi_1 \systemstep[l_S] \Pi_1'$.
We show that there is some $\Pi_2'$ such that $\Pi_1' \lessnd \Pi_2'$ and $\Pi_2 \systemstep[l_S] \Pi_2'$.
We proceed by case analysis of the system step.
\begin{description}
  \item[\textsc{Internal}]
    Let $\Pi_1' = \subst{\Pi_1}{L}{E}$, where $\Pi_1(L) \ctrlstep[\iota] E$.
    Then as $\Pi_1(L) \lessnd \Pi_2(L)$, by Lemma \ref{lem:internal-control-lifting} there is some $E'$ such that $E \lessnd E'$ and $\Pi_2(L) \ctrlstep[\iota] E'$.
    Therefore $\subst{\Pi_2}{L}{E'}$ suffices.
  \item[\textsc{Synchronized Internal}]
    Suppose that $\Pi_1(L) \ctrlstep[\iotasync] \Pi_1'(L)$ for each $L \in \Locations$.
    Then as $\Pi_1(L) \lessnd \Pi_2(L)$ for each $L$, by Lemma \ref{lem:internal-sync-control-lifting} there is some $\Pi_2'$ such that $\Pi_1'(L) \lessnd \Pi_2'(L)$ and $\Pi_2(L) \systemstep[\iotasync] \Pi_2'(L)$ for each $L$, as desired.
  \item[\textsc{Comm}]
    Suppose that $\Pi_1(L_1) \ctrlsteploc[v \sendsto L_2]{L_1} E_1$ and $\Pi_1(L_2) \ctrlsteploc[L_1;v \sendsto]{L_2} E_2$.
    Let $E_1'$ and $E_2'$ be the control expressions yielded by applying Lemma \ref{lem:send-control-lifting}.
    Then $\subst*{\Pi_2}{{L_1}{E_1'}{L_2}{E_2'}}$ satisfies the theorem.
  \item[\textsc{Choice}]
    Analogous to the \textsc{Comm} case.
  \item[\textsc{CommTy}]
    Analogous to the \textsc{Comm} case.
\end{description}