\subsection{Proof of Theorem \ref{thm:lifting-and-lowering}}
\begin{lem}[Choice Control Step Lowering]
\label{lem:choice-control-lowering}
For every choreography $C$ and distinct locations $L_1 \neq L_2$, if
\begin{itemize}
  \item $E_{1 \leq} \ctrlsteploc[d \sendsto L_2]{L_1} E_{1 \leq}'$,
  \item $\epp{C}{L_1} \lessnd E_{1 \leq}$,
  \item $E_{2 \leq} \ctrlsteploc[L_1;d \sendsto]{L_2} E_{2 \leq}'$, and
  \item $\epp{C}{L_2} \lessnd E_{2 \leq}$,
\end{itemize}
then there is some $E_1'$ and $E_2'$ such that
\begin{itemize}
  \item $\epp{C}{L_1} \ctrlsteploc[d \sendsto L_2]{L_1} E_1'$,
  \item $E_1' \lessnd E_{1 \leq}'$,
  \item $\epp{C}{L_2} \ctrlsteploc[L_1;d \sendsto]{L_2} E_2'$, and
  \item $E_2' \lessnd E_{2 \leq}'$.
\end{itemize}
\end{lem}
We proceed by induction on the choreography $C$ and the control steps.
\begin{description}
  \item[--]
    Let $C = C_1~C_2$.
    In this case, the control steps may either be \textsc{AppFun} or \textsc{AppArg}.
    We can apply induction separately on the function or argument, depending on the step, and keep the other part of the control expression the same.
  \item[--]
    Let $C = C' \ChorSend[L_a] L_b$.
    Because the control steps are synchronizations, neither $L_b = L_1$ nor $L_b = L_2$.
    In the case that $L_a = L_1$, the first control step must be \textsc{Send}.
    We can apply induction to $C'$ to yield some $E_1'$ and $E_2'$, where then $\SendTo{E_1'}{L_b}$ and $E_2'$ are satisfactory.
    The case when $L_a = L_2$ is symmetric.
    If both $L_a \neq L_1$ and $L_a \neq L_2$, we can directly apply induction to $C$.
  \item[--]
    Let $C = \syncs{L_a}{d}{L_b} \seq C'$.
    Because both the left and right case must be present in order to take an \textsc{AllowChoiceL} or \textsc{AllowChoiceR} step, we must have that $L_a \neq L_1, L_2$ and $L_b \neq L_1, L_2$.
    Therefore we can directly apply induction to $C'$ to yield the required $E_1'$ and $E_2'$.
  \item[--]
    Let $C = \ITE{C}{C_1}{C_2}$.
  \item[--]
    Let $C = C'~t$.
    In this case, the control steps must be \textsc{AppTFun}.
    Therefore we can apply induction to $C'$ to yield $E_1'$ and $E_2'$, with $E_1'~t$ and $E_2'~t$ sufficing for this case.
  \item[--]
    Let $C = \LetIn{L'.x \ty t_e}{C_1}{C_2}$.
    \begin{description}
      \item[--]
        Suppose that $L_1, L_2 \neq L'$.
        Then the control steps must both be \textsc{Seq}, and we can apply induction to $C_1$ to satisfy this case.
      \item[--]
        Suppose that $L_1 = L'$ and $L_2 \neq L'$.
        Then the control steps must be \textsc{LetRet} and \textsc{Seq}, respectively.
        In this case we can apply induction to $C_1$ to satisfy this case.
      \item[--]
        The case when $L_1 \neq L'$ and $L_2 = L'$ is symmetric to the previous.
    \end{description}
  \item[--]
    Let $C = \LetTellIn{\alpha}{L'}{C_1 \knd *_\ell}{\rho}{C_2}$.
    \begin{description}
      \item[--]
        Suppose that $L' = L_1 \in \rho$, and $L_2 \in \rho$.
      \item[--]
        Suppose that $L' = L_1 \in \rho$, and $L_2 \notin \rho$.
      \item[--]
        Suppose that $L' \neq L_1 \in \rho$, and $L_2 \in \rho$.
      \item[--]
        Suppose that $L' \neq L_1 \in \rho$, and $L_2 \notin \rho$.
      \item[--]
        Suppose that $L' = L_1 \notin \rho$, and $L_2 \in \rho$.
      \item[--]
        Suppose that $L' = L_1 \notin \rho$, and $L_2 \notin \rho$.
      \item[--]
        Suppose that $L' \neq L_1 \notin \rho$, and $L_2 \in \rho$.
      \item[--]
        Suppose that $L' \neq L_1 \notin \rho$, and $L_2 \notin \rho$.
    \end{description}
  \item[--]
    Let $C = \LetTellIn{\alpha}{L'}{C_1 \knd *_e}{\rho}{C_2}$.
\end{description}

\label{sec:lowering-proof}
\paragraph{Proof of \textsc{Lowering}}
Suppose that $\Pi_1 \lessnd \Pi_2$, $\Pi_2 \systemstep[l_S] \Pi_2'$, and $\Pi_1 = \epp{C}{\Locations}$.
We proceed by case analysis of the system step.
\begin{description}
  \item[\textsc{Internal}]
    Analogous to the corresponding case in the \textsc{Lifting} proof.
  \item[\textsc{Synchronized Internal}]
    Analogous to the corresponding case in the \textsc{Lifting} proof.
  \item[\textsc{Comm}]
    Analogous to the corresponding case in the \textsc{Lifting} proof.
  \item[\textsc{Choice}]
    Suppose that $\Pi_2(L_1) \ctrlsteploc[d \sendsto L_2]{L_1} E_1$ and $\Pi_2(L_2) \ctrlsteploc[L_1;d \sendsto]{L_2} E_2$.
    Let $E_1'$ and $E_2'$ be the control expressions yielded by applying Lemma \ref{lem:choice-control-lowering}.
    Then $\subst*{\Pi_1}{{L_1}{E_1'}{L_2}{E_2'}}$ satisfies the theorem.
  \item[\textsc{CommTy}]
  Analogous to the corresponding case in the \textsc{Lifting} proof.
\end{description}
