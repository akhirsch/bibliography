\subsection{Proof of Theorem \ref{thm:lifting-and-lowering}}

\begin{mathparpagebreakable}
  \ctxLabel{\hole} = \hole
  \and
  \ctxLabel{\eta;E} = \text{Fun}(\ctxLabel{\eta})
  \and
  \ctxLabel{E;\eta} = \text{Arg}(\ctxLabel{\eta})
  \and
  \ctxLabel{\eta;t} = \text{Fun}(\ctxLabel{\eta})
  \and
  \ctxLabel{\SendTo{\eta}{L}} = \text{Arg}(\ctxLabel{\eta})
  \and
  \ctxLabel{\CtrlITE{\eta}{E_1}{E_2}} = \text{Arg}(\ctxLabel{\eta})
  \and
  \ctxLabel{\CtrlLetIn{\Ret{x}}{\eta}{E}} = \text{Arg}(\ctxLabel{\eta})
  \and
  \ctxLabel{\LetSendIn{\alpha \knd \kappa}{\eta}{\rho}{E}} = \text{Arg}(\ctxLabel{\eta})
\end{mathparpagebreakable}

\begin{lem}[Choice Control Step Lowering]
\label{lem:choice-control-lowering}
For every choreography $C$, evaluation contexts $\eta_1$ and $\eta_2$, and distinct locations $L_1 \neq L_2$, if
\begin{itemize}
  \item $\eta_1[E_1] \ctrlreduceloc[d \sendsto L_2]{L_1} \eta_1[E_1']$,
  \item $\epp{C}{L_1} \lessnd \eta_1[E_1]$,
  \item $\eta_2[E_2] \ctrlreduceloc[L_1;d \sendsto]{L_2} \eta_2[E_2']$,
  \item $\epp{C}{L_2} \lessnd \eta_2[E_2]$, and
  \item $\ctxLabel{\eta_1} = \ctxLabel{\eta_2}$
\end{itemize}
then there is some $E_1''$ and $E_2''$ such that
\begin{itemize}
  \item $\epp{C}{L_1} \ctrlsteploc[d \sendsto L_2]{L_1} E_1''$,
  \item $E_1'' \lessnd \eta_1[E_1']$,
  \item $\epp{C}{L_2} \ctrlsteploc[L_1;d \sendsto]{L_2} E_2''$, and
  \item $E_2'' \lessnd \eta_2[E_2']$.
\end{itemize}
\end{lem}
We proceed by induction on the choreography $C$ and the evaluation contexts $\eta_1$ and $\eta_2$.
\begin{description}
  \item[--]
    If $C$ is of the form $X$, $\ell.e$ or $\Fun{F}{X \ty \tau}{C}$ then the reduction steps are impossible because either there is no matching evaluation context, or the label of the step is incorrect.
  \item[--]
    Let $C = C_1~C_2$.
    In this case, the evaluation contexts are either $\eta_1 = \eta_1'~E_1$ and $\eta_2 = \eta_2'~E_2$, or $\eta_1 = E_1~\eta_1'$ and $\eta_2 = E_2~\eta_2'$.
    In the former case, we can apply induction to $C_1$, $\eta_1'$, and $\eta_2'$ to yield some $E_1''$ and $E_2''$.
    Then $E_1''~E_1$ and $E_2''~E_2$ satisfy the conclusion.
    In the latter case, we can apply induction to $C_2$, $\eta_1'$, and $\eta_2'$ to yield some $E_1''$ and $E_2''$, for which $E_1~E_1''$ and $E_2~E_2''$ satisfy the conclusion.
  \item[--]
    Let $C = C' \ChorSend[\ell_1] \ell_2$.
    \begin{description}
      \item[--]
      Suppose that $L_1 = \ell_1$ and $L_2 = \ell_2$, so that $C$ projects to a send for $L_1$ and to a sequence for $L_2$. 
      Then the evaluation context for $L_1$ must be of the form $\eta_1 = \SendTo{\eta_1'}{L_2}$.
      We can apply induction with $C'$, $\eta_1'$, and $\eta_2$ to yield $E_1''$ and $E_2''$.
      Then the expressions $E_1'' \SendTo{E_1''}{L_2}$ and $E_2''$ suffice.
      \item[--]
      The case when $L_1 = \ell_2$ and $L_2 = \ell_1$ is symmetric.
      \item[--]
      Suppose that $L_1, L_2 \neq \ell_1, \ell_2$.
      Then we can directly apply induction on $C'$, $\eta_1$, and $\eta_2$.
    \end{description}
  \item[--]
    Let $C = \syncs{\ell_1}{d}{\ell_2} \seq C'$.
    Then necessarily $\ell_1 = L_1$, $\ell_2 = L_2$, $\eta_1 = \eta_2 = \hole$, $E_1 = \ChooseFor{d}{L_2}{E_1'}$, and $E_2 = \AllowOneChoice{L_1}{d}{E_2'}$, where $\epp{C'}{L_1} \lessnd E_1'$ and $\epp{C'}{L_2} \lessnd E_2'$.
    Therefore $\epp{C'}{L_1}$ and $\epp{C'}{L_2}$ are sufficient.
  \item[--]
    Let $C = \ITE{C'}{C_T}{C_F}$.
    \begin{description}
      \item[--]
      Suppose that $L_1, L_2 \neq \ell$.
      Then the evaluation contexts must be of the form $\eta_1 = \eta_1' \CtrlSeq E_1$ and $\eta_2 = \eta_2' \CtrlSeq E_2$, where $\epp{C_T}{L_1} \ctrlmerge \epp{C_F}{L_1} \lessnd E_1$ and $\epp{C_T}{L_2} \ctrlmerge \epp{C_F}{L_2} \lessnd E_2$.
      We can apply induction with $C'$, $\eta_1'$, and $\eta_2'$ to yield $E_1''$ and $E_2''$.
      Then the expressions $E_1'' \CtrlSeq E_1$ and $E_2' \CtrlSeq E_2$ suffice.
      \item[--]
      Suppose that $L_1 = \ell$.
      Then the evaluation contexts must be of the form $\eta_1 = \CtrlITE{\eta_1'}{E_T}{E_F}$ and $\eta_2 = \eta_2' \CtrlSeq E$, where $\epp{C_T}{L_1} \lessnd E_T$, $\epp{C_F}{L_1} \lessnd E_F$, and $\epp{C_T}{L_2} \ctrlmerge \epp{C_F}{L_2} \lessnd E$.
      We can apply induction with $C'$, $\eta_1'$, and $\eta_2'$ to yield $E_1''$ and $E_2''$.
      Then the expressions $\CtrlITE{E_1''}{E_T}{E_2'}$ and $E_2' \CtrlSeq \epp{C_T}{L_2} \ctrlmerge \epp{C_F}{L_2}$ suffice.
      \item[--]
      The case when $L_2 = \ell$ is symmetric.
    \end{description}
  \item[--]
    Let $C = C'~t$.
    In this case, the evaluation contexts must be of the form $\eta_1 = \eta_1'~t$ and $\eta_2 = \eta_2'~t$.
    We can apply induction with $C'$, $\eta_1'$, and $\eta_2'$ to yield $E_1''$ and $E_2''$.
    Then the expressions $E_1''~t$ and $E_2''~t$ suffice.
  \item[--]
    Let $C = \LetIn{\ell.x}{C_1}{C_2}$.
    \begin{description}
      \item[--]
        Suppose that $L_1, L_2 \neq \ell$, so the choreography projects to a sequence for both locations.
        Then the evaluation contexts must be of the form $\eta_1 = \eta_1' \CtrlSeq E_1$ and $\eta_2 = \eta_2' \CtrlSeq E_2$, so we can apply induction to $C_1$, $\eta_1'$, and $\eta_2'$ to satisfy this case.
      \item[--]
        Suppose that $L_1 = \ell$ and $L_2 \neq \ell$, so $C$ projects to a let-return for $L_1$, and to a sequence for $L_2$.
        Then the evaluation contexts must be of the form $\eta_1 = \CtrlLetIn{\Ret{x}}{\eta_1'}{E_1}$ and $\eta_2 = \eta_2' \CtrlSeq E_2$.
        We can apply induction to $C_1$, $\eta_1'$, and $\eta_2'$ to yield $E_1''$ and $E_2''$.
        Then the expressions $\CtrlLetIn{\Ret{x}}{E_1''}{E_1}$ and $E_2'' \CtrlSeq E_2$ are sufficient for this case.
      \item[--]
        The case when $L_1 \neq \ell$ and $L_2 = \ell$ is symmetric.
    \end{description}
  \item[--]
    Let $C = \LetTellIn{\alpha}{\ell}{C_1 \knd *_\ell}{\rho}{C_2}$.
    \begin{description}
      \item[--]
        Suppose that $L_1 = \ell$ and $L_2 \in \rho$.
        Then for $L_1$, $C$ projects to
        \begin{align*}
          &\LetSendIn{\alpha}{\epp{C_1}{L_1}}{\rho}{~}\\
          &~\AmI{\alpha}{\epp{\subst{C_2}{\alpha}{L}}{L}}{\epp{C_2}{L_1}}
        \end{align*}
        and for $L_2$, $C$ projects to
        \begin{align*}
          &\epp{C_1}{L} \CtrlSeq \LetRecvIn{\alpha}{\ell}{~} \\
          &~\AmI{\alpha}{\epp{\subst{C_2}{\alpha}{L}}{L}}{\epp{C_2}{L}}
        \end{align*}
        Thus the evaluation contexts must be of the form $\eta_1 = \LetSendIn{\alpha}{\eta_1'}{\rho}{E_1}$ and $\eta_2 = \eta_2' \CtrlSeq E_2$, and we can apply induction to $C_1$, $\eta_1'$, and $\eta_2'$.
      \item[--]
        Suppose that $L_1 = \ell$ and $L_2 \notin \rho$.
        Then for $L_2$, $C$ now projects to $\epp{C_1}{L_2} \CtrlSeq \epp{C_2}{L_2}$
        and its evaluation context is of the form $\eta_2 = \eta_2' \CtrlSeq E_2$, so we can again apply induction to $C_1$, $\eta_1'$, and $\eta_2'$.
      \item[--]
        Now suppose that $L_1, L_2 \notin \{\ell\} \cup \rho$.
        For $L_1$, $C$ projects to $\epp{C_1}{L_1} \CtrlSeq \epp{C_2}{L_1}$, and symmetrically for $L_2$.
        The evaluation contexts are of the form $\eta_1 = \eta_1' \CtrlSeq E_1$ and $\eta_2 = \eta_2' \CtrlSeq E_2$, so we can again apply induction to $C_1$, $\eta_1'$, and $\eta_2'$.
      \item[--]
        The other cases are symmetric.
    \end{description}
  \item[--]
    Let $C = \LetTellIn{\alpha \knd *_e}{\ell}{C_1}{\rho}{C_2}$.
    \begin{description}
      \item[--]
        Suppose that $L_1 = \ell$ and $L_2 \in \rho$.
        Then for $L_1$, $C$ projects to $\LetSendIn{\alpha}{\epp{C_1}{L_1}}{\rho}{\epp{C_2}{L_1}}$, and for $L_2$, $C$ projects to $\epp{C_1}{L} \CtrlSeq \LetRecvIn{\alpha}{\ell}{\epp{C_2}{L}}$.
        Then the evaluation contexts must be of the form $\eta_1 = \LetSendIn{\alpha}{\eta_1'}{\rho}{E_1}$ and $\eta_2 = \eta_2' \CtrlSeq E_2$, so we can apply induction to $C_1$, $\eta_1'$, and $\eta_2'$.
      \item[--]
        Suppose that $L_1 = \ell$ and $L_2 \notin \rho$.
        Then for $L_2$, $C$ now projects to $\epp{C_1}{L_2} \CtrlSeq \epp{C_2}{L_2}$
        and its evaluation context is of the form $\eta_2 = \eta_2' \CtrlSeq E_2$, so we can again apply induction to $C_1$, $\eta_1'$, and $\eta_2'$.
      \item[--]
        Now suppose that $L_1, L_2 \notin \{\ell\} \cup \rho$.
        For $L_1$, $C$ projects to $\epp{C_1}{L_1} \CtrlSeq \epp{C_2}{L_1}$, and symmetrically for $L_2$.
        The evaluation contexts are of the form $\eta_1 = \eta_1' \CtrlSeq E_1$ and $\eta_2 = \eta_2' \CtrlSeq E_2$, so we can again apply induction to $C_1$, $\eta_1'$, and $\eta_2'$.
      \item[--]
        The other cases are symmetric.
    \end{description}
\end{description}
