\section{Endpoint Projection}
\label{sec:endpoint-projection}
As is common for choreographic languages, our compilation procedure takes the  form of an \emph{endpoint projection} operation.
For each location participating in a choreography, this procedure extracts the program that the specified location will execute in the form of a \emph{control language}.
The extracted control programs can execute local computations, and perform send and receive operations.
A choreography, even when well-typed, may fail to project for a specific location.
If the choreography can be successfully projected for all participants, then we can compose them in parallel to provide an interpretation for the entire program.
Here we explain our control language, the endpoint projection procedure, and show that it preserves and reflects the semantics of choreographies.

\subsection{Control Language}
\label{sec:control-lang}

\todo purpose and design of control language

\todo explanation of AmI and reason it's needed

\begin{figure}
  \begin{syntax}
    \category[Control Expression]{E}
    \alternative{X} \alternative{\CtrlFail} \alternative{\Ret{e}} \alternative{\CtrlSeq{E_1}{E_2}} \alternative{\lambda X.E} \alternative{\mu X.E} \alternative{\ctrlapp{E_1}{E_2}} \\
    \alternative{\SendTo{E}{\ell}} \alternative{\RecvFrom{\ell}} \\
    \alternative{\ChooseFor{d}{\ell}{E}} \alternative{\AllowChoice{\ell}{E_1}{E_2}} \\
    \alternative{\ifthen{E}{E_1}{E_2}} \\
    \alternative{\Lambda \alpha.E} \alternative{\ctrlapp{E}{t}} \\
    \alternative{\LetIn{\Ret{x}}{E_1}{E_2}}\\
    \alternative{\LetSendIn{\alpha \knd *_\ell}{E_1}{\rho}{E_2}} \alternative{\LetRecvIn{\alpha \knd *_\ell}{\ell}{E}}\\
    \alternative{\LetSendIn{\alpha \knd *_e}{E_1}{\rho}{E_2}} \alternative{\LetRecvIn{\alpha \knd *_e}{\ell}{E}}\\
    \alternative{\AmI{\ell}{E_1}{E_2}}

    \category[Systems]{\Pi}
    \alternative{L_1 \triangleright E_1 \;\|\; \ldots \;\|\; L_n \triangleright E_n}
  \end{syntax}

  \caption{Control Expressions}
  \label{fig:control-lang-syntax}
\end{figure}

\paragraph{Operational Semantics}
\label{sec:control-lang-semantics}

\begin{figure}
  \begin{mathparpagebreakable}
      \infer[RetE]{e_1 \localstep e_2}
      {\Ret{e_1} \ctrlstep[\iota] \Ret{e_2}} \and

      \infer[SendV]{\text{Value}(V)}
      {\SendTo{V}{L} \ctrlstep[V \sendsto L] \CtrlFail} \and

      \infer[RecvV]{\text{Value}(V)}
      {\RecvFrom{L} \ctrlstep[L;V \sendsto] V} \and

      \infer[Choose]{}
      {\ChooseFor{d}{L}{E} \ctrlstep[d \sendsto L] E} \and

      \infer[AllowChoiceL]{}
      {\AllowChoice{L}{E_1}{E_2} \ctrlstep[L;\Left \sendsto] E_1} \and

      \infer[AllowChoiceR]{}
      {\AllowChoice{L}{E_1}{E_2} \ctrlstep[L;\Right \sendsto] E_2} \and

      \infer[AmIL]{}
      {\AmI{L}{E_1}{E_2} \ctrlstep[L] E_1} \and

      \infer[AmIR]{L \neq L'}
      {\AmI{L}{E_1}{E_2} \ctrlstep[L'] E_2} \and
  \end{mathparpagebreakable}
  \caption{Control Language Semantics}
  \label{fig:control-lang-sem}
\end{figure}

%%% Local Variables:
%%% mode: latex
%%% TeX-master: "conference"
%%% End:

\subsection{Endpoint Projection Procedure}
\label{sec:epp-defined}
\todo Explain epp. Form of compilation that gives each location their part of the choreography. Say that epp is partial for us because control program merging is partial. Note that merging is used in if-then-else projection.

\todo Explain epp. for systems
\begin{mathpar}
  \epp{C}{\Locations} = L \mapsto \epp{C}{L}
\end{mathpar}

\subsection{Properties of Endpoint Projection}
\label{sec:proj-correct}
\todo talk about less-nondeterminism relation

The less-nondeterminism relation can then be extended pointwise to systems.
That is,
\begin{mathpar}
  \Pi_1 \lessnd \Pi_2 \defeq \forall L \in \Locations,~ \Pi_1(L) \lessnd \Pi_2(L).
\end{mathpar}
The following theorem explains the utility of $\lessnd$ in relation to the system semantics.
Specifically, it says that (1) a more-nondeterministic system can always take the same steps as a less-nondeterministic system, and (2) a less-nondeterministic system can always take the same steps as a more-nondeterministic system so long as the less-nondeterministic system results from projecting a choreography.

\begin{thm}[Lifting and Lowering System Steps Across $\lessnd$]
  \label{thm:lifting-and-lowering}
  If $\Pi_1 \lessnd \Pi_2$, then the following are both true:
  \begin{itemize}
    \item[\textsc{Lifting}] If $\Pi_1 \systemstep[l_S] \Pi_1'$, then there is some $\Pi_2'$ such that $\Pi_1' \lessnd \Pi_2'$ and $\Pi_2 \systemstep[l_S] \Pi_2'$.
    \item[\textsc{Lowering}] If $\Pi_2 \systemstep[l_S] \Pi_2'$ and $\Pi_1 = \epp{C}{\Locations}$, then there is some $\Pi_1'$ such that $\Pi_1' \lessnd \Pi_2'$ and $\Pi_1 \systemstep[l_S] \Pi_1'$.
  \end{itemize}
\end{thm}

\todo theorem that is actually wanted

\todo extent of what is proven by agda

\todo proof of main theorem or intuition and link to appendix proof

%%% Local Variables:
%%% mode: latex
%%% TeX-master: "conference"
%%% End:

%%% Local Variables:
%%% mode: latex
%%% TeX-master: "conference"
%%% End: