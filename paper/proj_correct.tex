\subsection{Properties of Endpoint Projection}
\label{sec:proj-correct}
\todo talk about less-nondeterminism relation

The less-nondeterminism relation can then be extended pointwise to systems.
That is,
\begin{mathpar}
  \Pi_1 \lessnd \Pi_2 \defeq \forall L \in \Locations,~ \Pi_1(L) \lessnd \Pi_2(L).
\end{mathpar}
The following theorem explains the utility of $\lessnd$ in relation to the system semantics.
Specifically, it says that (1) a more-nondeterministic system can always take the same steps as a less-nondeterministic system, and (2) a less-nondeterministic system can always take the same steps as a more-nondeterministic system so long as the less-nondeterministic system results from projecting a choreography.

\begin{thm}[Lifting and Lowering System Steps Across $\lessnd$]
  \label{thm:lifting-and-lowering}
  If $\Pi_1 \lessnd \Pi_2$, then the following are both true:
  \begin{itemize}
    \item[\textsc{Lifting}] If $\Pi_1 \systemstep[l_S] \Pi_1'$, then there is some $\Pi_2'$ such that $\Pi_1' \lessnd \Pi_2'$ and $\Pi_2 \systemstep[l_S] \Pi_2'$.
    \item[\textsc{Lowering}] If $\Pi_2 \systemstep[l_S] \Pi_2'$ and $\Pi_1 = \epp{C}{\Locations}$, then there is some $\Pi_1'$ such that $\Pi_1' \lessnd \Pi_2'$ and $\Pi_1 \systemstep[l_S] \Pi_1'$.
  \end{itemize}
\end{thm}

\todo theorem that is actually wanted

\todo extent of what is proven by agda

\todo proof of main theorem or intuition and link to appendix proof

%%% Local Variables:
%%% mode: latex
%%% TeX-master: "conference"
%%% End: