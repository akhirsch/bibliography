% !TeX root = main.tex
\section{Related Work}
\label{sec:related-work}

As mentioned in Section~\ref{sec:introduction}, choreographic programming has seen a large amount of interest lately.
In particular, this paper fits in the emerging paradigm of \emph{functional} choreographic programming.
Process polymorphism is a technique within choreographic programming proposed by~\citet{GraversenHM24}.
Choreographic programming as a whole arises from concurrency theory and the study of $\pi$~calculi.
We discuss each of these in turn.

\subsection{Functional Choreographic Programming}
\label{sec:choreo-prog}

\subsection{Process Polymorphism}
\label{sec:process-poly}

\subsection{Higher-Order Communication}
\label{sec:higher-order-comm}