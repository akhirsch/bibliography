\section{Agda Development}
\label{sec:agda-dev}

In this paper we have presented \langname in a standard style with named variables.
However for ease of development, our Agda code uses a nameless style implemented with de~Bruijn indices.
As well, we use a specific representation for the local language which guarantees that the required admissible syntactic and typing rules hold, yet formally is not any less general.

\todo References specific to binding signatures

\ashley{How detailed to make it? I think this may already be too much?}
Specifically, we represent the syntax, kinding system, and type system of the local language, choreographic language, and control language using \emph{binding signatures}.
In this representation, types and terms are described via \emph{operations} in a similar manner to how they are presented in a Backus–Naur form.
Associated with each operator is an arity, quantifying the number of subexpressions, and for each subexpression a count of the number of variables bound in that subexpression.
For example, the basic $\lambda$-calculus has three forms of expressions: variables, $\lambda$-abstractions, and applications.
\begin{syntax}
	\category{e}
	\alternative{x}
	\alternative{\lambda x.e}
	\alternative{e_1~e_2}
\end{syntax}
Variables are naturally assumed to exist in this framework, so there is no operator associated with them.
The operator for $\lambda$-abstraction has arity 1, with the sole subexpression having 1 bound variable.
The operator for application has arity 2, where neither subexpression introduces any bound variables.

Given its binding signature, the syntax and type system associated with a language, as well as certain metafunctions such as substitution, can be automatically defined.
Using this representation allows us to prove standard syntactic and typing metatheorems for all languages at once, reducing boilerplate.
As well, since our choreographic and control languages are still parameterized by the choice of local language, our theorems are no-less general.

%%% Local Variables:
%%% mode: latex
%%% TeX-master: "conference"
%%% End: