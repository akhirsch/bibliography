\documentclass{article}

\usepackage[utf8]{inputenc}
\usepackage[T1]{fontenc}
\usepackage[letterpaper, margin=1in, bottom=1in]{geometry}
\usepackage[authoryear,sort,square]{natbib}
\usepackage{amsmath}
\usepackage{amssymb}
\usepackage{amsthm}
\usepackage{bbm}
\usepackage{bbold}
\usepackage{stmaryrd}
\usepackage{mathtools}
\usepackage{mathpartir}
\usepackage[dvipsnames]{xcolor}
\usepackage{latex-pl-syntax/pl-syntax}
\usepackage{xspace}
\usepackage{suffix}
\usepackage{turnstile}
\usepackage{multicol}

\usepackage[hidelinks]{hyperref}

% Notes
\newcommand{\uncertain}[1]{{\color{red} #1}\xspace}
\newcommand{\newcommenter}[3]{%
  \newcommand{#1}[1]{%
    \textcolor{#2}{\small\textsf{[{#3}: {##1}]}}%
  }%
}
\definecolor{darkgreen}{rgb}{0,0.7,0}
\newcommenter{\akh}{purple}{AKH}

% AMSTHM Setup
\newtheorem{thm}{Theorem}
\newtheorem{lem}{Lemma}
\newtheorem{cor}{Corollary}
\newtheorem{conj}{Conjecture}
\newtheorem{inv}{Invariant}
\theoremstyle{definition}
\newtheorem{defn}{Definition}

% Macros

%% Assumptions on expressions
\newcommand{\bool}{\mathbbm{2}}
\newcommand{\ett}{\textsf{true}}
\newcommand{\eff}{\textsf{false}}

%% Evaluation
\newcommand{\To}{\Rightarrow}

%% Substitution
\newcommand{\subst}[3]{#1~[#2 \mapsto #3]}

%% Equivalence
\newcommand{\cleq}{\preceq}
\newcommand{\ceq}{\equiv}

%% Choreography Language
\newcommand{\own}[2]{#1.#2}
\newcommand{\cdone}{\mathbb{0}}
\newcommand{\send}[3][]{#2\nolinebreak \mathrel{\xrightarrow{#1}}\nolinebreak #3}
\newcommand{\cif}[3]{\textsf{if}~#1 \mathrel{\textsf{then}}\nolinebreak #2 \mathrel{\textsf{else}}\nolinebreak #3}
\newcommand{\clet}[2]{\textsf{let}~#1\nolinebreak =\nolinebreak #2 \mathrel{\textsf{in}}}
\newcommand{\copen}[2]{\textsf{open}_{#1}~#2}
\newcommand{\idchan}{\textsf{id\_chan}}
\newcommand{\match}{\ensuremath{\mathsf{match}\xspace}}
\newcommand{\with}{\ensuremath{\mathsf{with}}\xspace}
\newcommand{\End}{\ensuremath{\mathsf{end}}\xspace}
\newcommand{\inl}{\ensuremath{\mathsf{inl}}\xspace}
\newcommand{\inr}{\ensuremath{\mathsf{inr}}\xspace}

\newcommand{\matchsum}[5]{%
  \begin{array}{l}%
    \match~#1~\with\\%
    |~\inl(#2) \mathrel{\Rightarrow} #3\\%
    |~\inr(#4) \mathrel{\Rightarrow} #5\\%
    \End%
  \end{array}%
}
\WithSuffix\newcommand\matchsum*[5]{\match~#1~\with\quad|\, \inl(#2)\mathrel{\Rightarrow}#3\quad|\, \inr(#4) \mathrel{\Rightarrow} #5 \quad\End}
\newcommand{\matchpair}[4]{%
  \begin{array}{l}%
    \match~#1~\with\\%
    |~(#2,~#3) \mathrel{\Rightarrow} #4\\%
    \End%
  \end{array}%
}
\WithSuffix\newcommand\matchpair*[4]{\match~#1~\with\quad|\, (#2,#3)\mathrel{\Rightarrow} #4 \quad\End}


\newcommand{\eppctxt}[2]{\llparenthesis #1 \rrparenthesis_{#2}}

%% IR

   

%% Type Judgments
\newcommand{\proves}{\vdash}
%\newcommand{\proves}[1][]{\sststile{}{#1}}

%% BL_S Commands
\newcommand{\says}{\ensuremath{\mathrel{\textsf{says}}}}
\newcommand{\claims}{\ensuremath{\mathrel{\textsf{claims}}}}

%% Translations
\newcommand{\transb}[1]{\left\llbracket #1 \right\rrbracket}

\newcommand{\prep}[4]{\mathsf{prepare}_{#1 \xrightarrow{#2} #3}\left(#4\right)}

\bibliographystyle{plainnat}

\begin{document}


\section{Overview}
\label{sec:overview}

Modal logic has been used in the security world for reasoning about distributed systems--in the form of authorization logic--for a long time.
Other work, such as the work on ML5~\citep{MurphyCH07}, has used modal logic as the basis of type systems for distributed programs.

These two threads of research take very different views of how modal logic models distributed programming.
Authorization logic~\citep{HirschACAT20-TR,HirschC13,GenoveseGR12,SchneiderWS11,GenoveseGGP11,GargP10,GargA08,Garg08,Abadi08,GargP06,Abadi03} uses multi-modal logic, so that each principal is represented as a modality, traditionally written "p says."
Since each principal is a reasoner, a propositions-as-types point-of-view regards each principal as a center of computation, such as a thread, processor, or computer in a distributed system.
ML5~\citep{MurphyCH07} and related works~\citep{JiaW04,AhmedJW03}, in contrast, use a hybrid logic form of S5, a standard modal logic.
This allows them to model each center of computation as a world in a Kripke semantics for S5, and to model communication via a monotonicity condition on worlds.

While these threads of research differ significantly in their logical form, they have a similar view of distributed programming.
In both, we reason about distributed programs from outside of the system and direct communication between participants.
This creates a programming style similar to that used for describing distributed protocols e.g., in the cryptography literature.
This in contrast to process calculi like $\pi$-calculus, CCS, and join calculus, which write programs as parallel compositions of code for each participant.

In the world of calculi for distributed programming, choreographies~\citep{Cruz-FilipeM17,Cruz-FilipeMP18,CarboneMS14,Montesi13,ZongyanXCH07,CarboneM13} take this same top-down view of distributed programming.
Choreographies not only allow a programmer to write programs in a top-down manner, but then give the ability to \emph{project} process calculus code for each participant.
This is referred to as \emph{endpoint projection}.
This process calculus code is guaranteed by construction to be deadlock free.

Choreographies are designed to lift some \emph{expressions}---which are purely local and which return values---to a process by describing how processes share and use values calculated from those expressions.
This tells us something about the limits of deadlock-freedom: if expressions can get stuck, then so can choreographies that use those expressions.
Previous works on choreographies either provide an explicit, terminating, expression language~\citep{Cruz-FilipeM17,CarboneMS14,ZongyanXCH07}; assume that expressions terminate~\citep{Cruz-FilipeMP18}; or provide no guarantee of progress~\citep{CarboneM13}.

Usually, progress is guaranteed by a type system.
While type systems for choreographies have been studied before~\citep{CarboneM13,CarboneMS14}, these use \emph{session types}, which constrain a process to ensure that it follows the correct pattern of communication.
\akh{I'm pretty sure there's more type systems for choreographies papers.}
While this has been used by \citet{CarboneM13}, 

\section{A Multimodal Type System for Choreographies}
\label{sec:mult-type-syst-chor}

\subsection{Expressions}
\label{sec:expressions-mmc}

I assume we have an expression language with a typing system.
We assume that typing judgments for that language have the form $\Gamma \proves e : \tau$.

\subsection{Choreographies}
\label{sec:choreographies-mmc}

\paragraph{Syntax}
\begin{syntax}
  \abstractCategory[Choreography Variables]{X,Y,Z,\ldots}
  \abstractCategory[Thread Names]{p,q,r,\ldots}
  \category[Choreographies]{C}
    \alternative{p.e}
    \alternative{X}
    \alternative{\send{p.e}{q.x};~C}
    \alternative{\cif{p.e}{C_1}{C_2}}
    \alternative{\clet{X}{C_1}{C_2}}
\end{syntax}
  
\paragraph{Context Projection}
\noindent\begin{align*}
  \eppctxt{\cdot}{p} &= \cdot\\
  \eppctxt{x : p.\tau, \Gamma}{p} &= x : \tau, \eppctxt{\Gamma}{p}\\
  \eppctxt{x : q.\tau, \Gamma}{p} &= \eppctxt{\Gamma}{p}~\textrm{(where $p \neq q$)}
\end{align*}

\paragraph{Type System}
\noindent\begin{mathpar}
  \infer{\eppctxt{\Gamma}{p} \proves e : \tau }{\Gamma; \Delta \proves p.e : p.\tau}\and
  \infer{ }{\Gamma; \Delta, X : p.\tau \proves X : p.\tau} \and
  \infer{\eppctxt{\Gamma}{p} \proves e : \tau\\ \Gamma, x : q.\tau; \Delta \proves C : r.\sigma}{\Gamma; \Delta \proves \send{p.e}{q.x}; C : r.\sigma}\and
  \infer{\eppctxt{\Gamma}{p} \proves e : \bool\\ \Gamma; \Delta \proves C_1 : q.\tau\\ \Gamma; \Delta \proves C_2 : q.\tau}{\Gamma; \Delta \proves \cif{p.e}{C_1}{C_2} : q.\tau} \and
  \infer{\Gamma; \Delta, X : p.\tau \proves C_1 : p.\tau \\ \Gamma; \Delta, X : p.\tau \proves C_2 : q.\sigma}{\Gamma; \Delta \proves \clet{X}{C_1}{C_2} : q.\sigma}
\end{mathpar}

\paragraph{Substitution}

\noindent Substitution on expressions is given.

\begin{align*}
  \subst{p.e}{X}{C} &= p.e \\
  \subst{X}{X}{C} &= X \\
  \subst{Y}{X}{C} &= Y~\text{(assuming that $X \neq Y$)} \\
  \subst{(\send{\own{p}{e}}{\own{q}{x}};~C_1)}{X}{C_2} &= \send{\own{p}{e}}{\own{q}{x}};~(\subst{C_1}{X}{C_2}) \\
  \subst{(\cif{\own{p}{e}}{C_1}{C_2})}{X}{C_3} &= \cif{\own{p}{e}}{(\subst{C_1}{X}{C_3})}{(\subst{C_2}{X}{C_3})}\\
  \subst{(\clet{X}{C_1}{C_2})}{X}{C_3} &= \clet{X}{C_1}{C_2}\\
  \subst{(\clet{X}{C_1}{C_2})}{Y}{C_3} &= \clet{X}{(\subst{C_1}{Y}{C_3})}{(\subst{C_2}{Y}{C_3})}
\end{align*}

\begin{align*}
  \subst{p.e_1}{\own{p}{x}}{e_2} &= p.(\subst{e_1}{x}{e_2})\\
  \subst{p.e_1}{\own{q}{x}}{e_2} &= p.e_1~\textrm{(where $p \neq q$)}\\
  \subst{X}{\own{p}{x}}{e} &= X\\
  \subst{(\send{\own{p}{e_1}}{\own{p}{x}}; C)}{\own{p}{x}}{e_2} &= \send{\own{p}{(\subst{e_1}{x}{e_2})}}{\own{q}{x}}; C\\
  \subst{(\send{\own{p}{e_1}}{\own{q}{x}}; C)}{\own{p}{y}}{e_2} &= \send{\own{p}{(\subst{e_1}{y}{e_2})}}{\own{q}{x}}; (\subst{C}{\own{p}{y}}{e_2})\\
  \subst{(\send{\own{p}{e_1}}{\own{q}{x}}; C)}{\own{q}{x}}{e_2} &= \send{\own{p}{e_1}}{\own{q}{x}}; C\\
  \subst{(\send{\own{p}{e_1}}{\own{q}{x}}; C)}{\own{r}{y}}{e_2} &= \send{\own{p}{e_1}}{\own{q}{x}}; (\subst{C}{\own{r}{y}}{e_2})\\
  \subst{(\cif{\own{p}{e_1}}{C_1}{C_2})}{\own{p}{x}}{e_2} &= \cif{\own{p}{(\subst{e_1}{x}{e_2})}}{(\subst{C_1}{\own{p}{x}}{e_2})}{(\subst{C_2}{\own{p}{x}}{e_2})}\\
  \subst{(\cif{\own{p}{e_1}}{C_1}{C_2})}{\own{q}{x}}{e_2} &= \cif{\own{p}{e_1}}{(\subst{C_1}{\own{q}{x}}{e_2})}{(\subst{C_2}{\own{q}{x}}{e_2})}
\end{align*}

\paragraph{Equivalence}
\noindent\begin{mathpar}
  \infer{ }{C \ceq C} \and \infer{C_1 \ceq C_2}{C_2 \ceq C_1} \and \infer{C_1 \ceq C_2 \\ C_2 \ceq C_3}{C_1 \ceq C_3} \\
  \infer{C_1 \ceq C_2}{\send{\own{p}{e}}{\own{q}{x}}; C_1 \ceq \send{\own{p}{e}}{\own{q}{x}}; C_2} \and
  \infer{C_{11} \ceq C_{21}\\ C_{12} \ceq C_{22}}{\cif{\own{p}{e_1}}{C_{11}}{C_{12}} \ceq \cif{\own{p}{e_1}}{C_{21}}{C_{22}}} \and
  \infer{C_{11} \ceq C_{21}\\ C_{12} \ceq C_{22}}{\clet{X}{C_{11}}{C_{12}} \ceq \clet{X}{C_{21}}{C_{22}}} \\
  \infer{p \neq r\\ q \neq r\\ p \neq s\\ q \neq s}{\send{\own{p}{e_1}}{\own{q}{x}}; \send{\own{r}{e_2}}{\own{s}{y}}; C \ceq \send{\own{r}{e_2}}{\own{s}{y}}; \send{\own{p}{e_1}}{\own{q}{x}}; C} \and
  \infer{p \neq q\\ p \neq r}{\cif{\own{p}{e_1}}{\send{\own{q}{e_2}}{\own{r}{x}}; C_1}{\send{\own{q}{e_2}}{\own{r}{x}}; C_2} \ceq \send{\own{q}{e_2}}{\own{r}{x}}; \cif{\own{p}{e_1}}{C_1}{C_2}} \and
  \infer{p \neq q }{\cif{\own{p}{e_1}}{(\cif{\own{q}{e_2}}{C_1}{C_2})}{\cif{(\own{q}{e_2}}{C_3}{C_4})} \ceq \cif{\own{q}{e_2}}{(\cif{\own{p}{e_1}}{C_1}{C_3})}{(\cif{\own{p}{e_2}}{C_2}{C_4})}} \and
\end{mathpar}

\begin{thm}[Equivalence Preserves Typing]
  \label{thm:equiv-pres-types}
  If $\Gamma \proves C_1 : p.\tau$ and $C_1 \ceq C_2$, then $\Gamma \proves C_2 : p.\tau$
\end{thm}
\begin{proof}
  Proven in Coq.
\end{proof}

\begin{thm}[Equivalence is Stable under Substitution]
  \label{thm:equiv-stable-subst}
  If $C_1 \ceq C_2$, then $\subst{C_1}{\own{p}{x}}{e} \ceq \subst{C_2}{\own{p}{x}}{e}$ and $\subst{C_1}{X}{C} \ceq \subst{C_2}{X}{C}$.
\end{thm}
\begin{proof}
  Proven in Coq.
\end{proof}

\paragraph{Operational Semantics}


\subparagraph{Standard Presentation}
\noindent\begin{mathpar}
  \infer{e_1 \To e_2}{p.e_1 \To p.e_2} \and
  \infer{e_1 \To e_2}{\send{\own{p}{e_1}}{\own{q}{x}};~C \To \send{\own{p}{e_2}}{\own{q}{x}};~C} \and
  \infer{}{\send{\own{p}{v}}{\own{q}{x}}; C \Rightarrow \subst{C}{\own{q}{x}}{v}} \and
  \infer{e_1 \Rightarrow e_2}{\cif{\own{p}{e_1}}{C_2}{C_3} \To \cif{\own{p}{e_2}}{C_2}{C_3}} \and
  \infer{}{\cif{\own{p}{\ett}}{C_1}{C_2} \To C_1} \and
  \infer{}{\cif{\own{p}{\eff}}{C_1}{C_2} \To C_1} \and
  \infer{}{\clet{X}{C_1}{C_2} \To \subst{C_2}{X}{\clet{X}{C_1}{C_1}}} \and
  \infer{C_1 \ceq C_1'\\ C_1' \To C_2'\\ C_2' \ceq C_2}{C_1 \To C_2}
\end{mathpar}

This is the standard presentation.
However, it turns out to be very difficult to work with, because the equivalence breaks induction.
We might hope to make the last rule admissible, getting around this.
However, the last rule allows you to ``reach under'' the current top-level construct to allow steps from other threads to be interleaved.

\subparagraph{Presentation with Better Induction Principle}
In order to get around this, we present the operational semantics differently.
There are two difficulties here: only allowing operations from other threads, and ensuring that threads that are implicitly blocked from an if statement do not compute.
To see what goes wrong, consider the following two (non) reductions:
\noindent\begin{mathpar}
  \infer{e_2 \To e_3}{\send{\own{p}{e_1}}{\own{q}{x}}; \send{\own{p}{e_2}}{\own{r}{y}}; C \To \send{\own{p}{e_1}}{\own{q}{x}}; \send{\own{p}{e_3}}{\own{r}{y}}} \and
  \infer{e_2 \To e_4}{\cif{\own{p}{e_1}}{\send{\own{q}{e_2}}{\own{r}{x}}; C_1}{\send{\own{q}{e_3}}{\own{s}{y}}; C_2} \To \cif{\own{p}{e_1}}{\send{\own{q}{e_4}}{\own{r}{x}}; C_1}{\send{\own{q}{e_3}}{\own{s}{y}}; C_2}}
\end{mathpar}
In the first, $p$ is a single thread which does work on $e_2$, even though that thread is currently working on sending $e_1$ to $q$.
In the second, $q$ cannot know which of the two messages it is going to send, and so it should not be able to do work on either side.
(Intuitively, it is waiting for a message from $p$ to know which action to take.
We make that intuition explicit in the translation to $\pi$-calculus.)

To take care of the first problem, we annotate the operational semantics with a set of principals~$B$ which are currently blocked (and therefore cannot take action).
However, this is not enough in the second case, since $q$ is implicitly blocked by a decision.
In order to take care of this, we also annotate the operational semantics with a \emph{redex}, ensuring that we operate on the same expression, in the same place, on the same thread in both sides of the match statement.
This, combined with the set of blocked principals, is enough to ensure that implicitly-blocked threads do not progress.

This leads to the following operational semantics:

\begin{syntax}
  \category[Redeces]{R}
    \alternative{\text{Done}\, p\, e_1\, e_2}
    \alternative{\text{SendE}\, p\, e_1\, e_2\,q}
    \alternative{\text{SendV}\, p\, v\,q}
    \alternative{\text{IfE}\, p\, e_1\, e_2}
    \alternative{\text{IfTT}\, p}
    \alternative{\text{IfFF}\, p}
    \alternative{\text{Def}}
\end{syntax}

\noindent\begin{mathpar}
  \infer{e_1 \To e_2} {p.e_1 \xRightarrow{\text{Done}\,p\,e_1\,e_2,~\varnothing} p.e_2} \and
  \infer{p \notin B\\ q \notin B\\ e_1 \To e_2}{\send{\own{p}{e_1}}{\own{q}{x}}; C \xRightarrow{\text{SendE}\,p\,e_1\,e_2\,q,~B} \send{\own{p}{e_2}}{\own{q}{x}}; C} \and
  \infer{p \notin B\\\ q \notin B}{\send{\own{p}{v}}{\own{q}{x}}; C \xRightarrow{\text{SendV}\,p\,v\,q,~B} \subst{C}{\own{q}{x}}{v}} \and
  \infer{C_1 \xRightarrow{R,~B \cup \{p,q\}} C_2}{\send{\own{p}{e}}{\own{q}{x}}; C_1 \xRightarrow{R,~B} \send{\own{p}{e}}{\own{q}{x}}; C_2} \and
  \infer{p \notin B\\ e_1 \To e_2}{\cif{\own{p}{e_1}}{C_1}{C_2} \xRightarrow{\text{IfE}\,p\,e_1\,e_2,~B} \cif{\own{p}{e_2}}{C_1}{C_2}} \and
  \infer{p \notin B}{\cif{\own{p}{\ett}}{C_1}{C_2} \xRightarrow{\text{IfTT}\,p,~B} C_1} \and
  \infer{p \notin B}{\cif{\own{p}{\eff}}{C_1}{C_2} \xRightarrow{\text{IfFF}\,p,~B} C_2} \and
  \infer{C_1 \xRightarrow{R,~B \cup \{p\}} C_3\\ C_2 \xRightarrow{R,~B \cup \{p\}} C_4}{\cif{\own{p}{e}}{C_1}{C_2} \xRightarrow{R,~B} \cif{\own{p}{e}}{C_3}{C_4}} \and
  \infer{ }{\clet{X}{C_1}{C_2} \xRightarrow{\text{Def},~\varnothing} \subst{C_2}{X}{\clet{X}{C_1}{C_1}}}
\end{mathpar}


\begin{thm}[Simulation of Equivalence]
  \label{thm:simulation-equiv}
  If $C_1 \ceq C_2$ and $C_1 \xRightarrow{R,~B} C_1'$, then there is a $C_2'$ such that $C_2 \xRightarrow{R,~B} C_2'$ and $C_1' \ceq C_2'$.
\end{thm}
\begin{proof}
  Proven in Coq.
\end{proof}

\begin{thm}[Lifting Progress]
  \label{thm:lifting-progress}
  If $\proves e : \tau$ implies that there is some $e'$ such that $e \To e'$ for every expression $e$ and, for every value~$v$ such that $\Gamma \proves c : \bool$, $v$ is either $\ett$ or $\eff$, then for any choreography $C$, $; \proves C : p.\tau$ implies that there is some $R$ and $C'$ such that $C \xRightarrow{R,~\varnothing} C'$.
\end{thm}
\begin{proof}
  Proven in Coq.
\end{proof}

\begin{thm}[Lifting Preservation]
  \label{thm:lifting-preservation}
  If $\Gamma \proves e : \tau$ and $e \to e'$ implies $\Gamma \proves e' : \tau$, then $\Delta; \Gamma \proves C : p.\tau$ and $C \xRightarrow{R,~B} C'$ implies $\Delta; \Gamma \proves C' : p.\tau$.
\end{thm}
\begin{proof}
  Proven in Coq.
\end{proof}

\begin{thm}[Injection into Standard Presentation]
  If $C_1 \xRightarrow{R,~B} C_2$, then $C_1 \To C_2$.
\end{thm}
\begin{proof}
  Proven in Coq.
\end{proof}

\begin{thm}[Injection of Standard Presentation]
  If $C_1 \To C_2$ then there is an $R$ and $C_2'$ such that $C_2 \ceq C_2'$ and $C_1 \xRightarrow{R,~\varnothing} C_2'$.
\end{thm}
\begin{proof}
  Proven in Coq.
\end{proof}


\subsection{Endpoint Projection}

We have a couple of options for how to handle endpoint projection.
I explore a couple here, and note the strengths and weaknesses of each.
In particular, I will look at the difficulty of projecting the following simple choreography (adapted from~\citet{AschieriG20}):
$$\send{\own{\textsf{Alice}}{n}}{\own{\textsf{Bob}}{x}};~\send{\own{\textsf{Bob}}{\textsf{enc}(x)}}{\own{\textsf{Cathy}}{y}};~\send{\own{\textsf{Cathy}}{\textsf{enc}(y)}}{\own{\textsf{Alice}}{z}}; \own{\textsf{Alice}}{z}$$
This demonstrates the ability for choreographies to deal with cyclic communication, which appears to be hard in session types.

In any case, for any thread name $p$ we write $\transb{C}_p$ for $p$'s endpoint projection of $C$.
This is a process-calculus program representing $p$'s actions in $C$.

\subsubsection{Intuitionistic Session Types}
We project into the following $\pi$-calculus (taken from \citet{ToninhoCP12}):
\begin{syntax}
  \category[Processes]{P,Q,\ldots}
  \alternative{0}
  \alternative{P | Q}
  \alternative{(\nu y) P}
  \alternative{x\langle y \rangle.P}
  \alternative{x(y).P}
  \alternative{!x(y).P}
  \alternative{[y \leftrightarrow x]}
\end{syntax}
With the following type system:
\begin{mathpar}
  \infer{ }{\Gamma;x : A \proves [x \leftrightarrow z] :: z : A} \and
  \infer{\Gamma; \Delta, x : A \proves P :: z : B}{\Gamma, \Delta \proves z(x).P :: z : A \multimap B} \and
  \infer{\Gamma; \cdot \proves P :: x : A}{\Gamma; \cdot \proves !z(x).P :: z : !A} \and
  \infer{\Gamma, u : A; \Delta \proves P :: z : C}{\Gamma; \Delta, x : !A \proves P[u \mapsto x] :: z : C} \and
  \infer{\Gamma; \Delta \proves P :: y : A\\ \Gamma; \Delta', x : B \proves Q :: z : C}{\Gamma; \Delta, \Delta', x : A \multimap B \proves (\nu y)x\langle y \rangle.(P | Q) :: z : C} \and
  \infer{\Gamma, u : A; \Delta, x : A \proves P :: z : C}{\Gamma, u : A; \Delta \proves (\nu x)u\langle x\rangle.P :: z : C} \and
  \infer{\Gamma; \Delta \proves P :: x : A\\ \Gamma; \Delta', x : A \proves Q :: z : C}{\Gamma; \Delta, \Delta' \proves (\nu x)(P | Q) :: z : C} \and
  \infer{\Gamma; \cdot \proves P :: x : A\\ \Gamma, u : A; \Delta \proves Q :: z : C}{\Gamma; \Delta \proves (\nu u)(!u(x).P | Q) :: z : C}
\end{mathpar}

Now, for a typed choreography $\Gamma; \Delta \proves C : p.\tau$, we define four projections:
\begin{enumerate}
\item The \emph{leader type} projection, which will tell us the session type for $p$
\item The \emph{leader program} projection, which will be the EPP for $p$
\item The \emph{follower type} projection, which will tell us the session type for all other principals
\item The \emph{follower program} projection, which will be the EPP for all other principals
\end{enumerate}

The idea is that for each principal $q \neq p$, we will have
$$\proves \llbracket C \rrbracket^{\text{FP}}_q : \llbracket C \rrbracket^{\text{FT}}_q$$
and for $p$ we will have
$$ \cdot; \{x_q : \llbracket C \rrbracket^{\text{FT}}_q \mid q \neq p\} \proves \llbracket C \rrbracket^{\text{LP}}_p : \llbracket C \rrbracket^{\text{LT}}_p$$

I'm not sure that this will be possible.
In particular, follower processes can communicate among each other freely, and we don't have any record of that.
The idea is to use the left-$\multimap$ rule, but this may not be possible as defined.

Another possibility is to get rid of the idea of the leader and follower, and treat everyone as ``their own leader.''
However, this then would require tying a recursive knot, since everyone needs to be cut into everyone else at the same time.

Reading \emph{Par Means Parallel}~\citep{AschieriG20} suggests that this problem is fundamental whenever there is cyclic communication.
This lead to me considering the example above.
Intuitively, we want the endpoint projection of this program to look like:
$$
\begin{array}{ll}
    & m_{1}\langle n \rangle. m_{3}(z). m_{\textsf{out}}\langle z \rangle.0~\textcolor{gray}{\textit{-- Alice's program}}\\
  | & m_{1}(x). m_{2} \langle \textsf{enc}(x) \rangle.0~\textcolor{gray}{\textit{-- Bob's program}}\\
  | & m_{2}(y). m_{3} \langle \textsf{enc}(y) \rangle.0~\textcolor{gray}{\textit{-- Cathy's program}}
\end{array}
$$
However, this is impossible to typecheck in \citeauthor{ToninhoCP12}'s~\citeyear{ToninhoCP12} type system.
To see this, consider the rule
$$\infer*[right=$\multimap$L]{\Gamma; \Delta \proves P :: y : A\\ \Gamma; \Delta', x : B \proves P :: z : C}{\Gamma; \Delta, \Delta', x : A \multimap B \proves (\nu y)x\langle y \rangle.(P | Q)}$$
This is the only rule that type-checks message sending.
However, it requires $P$ and $Q$ to be in parallel \emph{before} the send.
Thus, endpoint projection cannot proceed in this method.

\subsubsection{Multiparty Session Types}

Another option is to use multiparty session types~\citep{ScalasY19}.
\akh{I should add more multiparty session types papers to the bibliography.}
In particular, let's consider the system from ``Less is More: Multiparty Session Types Revisited''~\citep{ScalasY19}.
This system is defined as follows:
\begin{syntax}
  \abstractCategory[Roles]{p,q,\ldots}
  \category[Channels]{c,d,\ldots} \alternative{x} \alternative{s[p]}
  \category[Processes]{P,Q,\cdots} \alternative{0} \alternative{P|Q} \alternative{(\nu s) P}
  \alternativeLine{c[q]\oplus m\langle d\rangle.P}
  \alternativeLine{c[q]\sum_{i \in I} m_i(x_i).P_i}
  \alternativeLine{\textsf{def}~D~\textsf{in}~P} \alternative{X \langle\vec{c}\rangle} \alternative{\textsf{err}}
\end{syntax}
In the syntax for channels, a channel is either a variable $x$ or a session with a role $s[p]$.
In the syntax for processes, $c[q]\oplus m\langle d\rangle$ represents sending a selection to role $q$ along $c$, while $c[q]\sum_{i \in I} m_i(x_i).P_i$ represents offering $q$ a choice along $c$.
In both cases, $m$ is a label determining which branch to proceed along, while $d$ is some data to send along with the selection, and $x_i$ is a variable to be bound to the received data.
When we only want to represent a send or a receive, we will write $c[q]\langle d \rangle$ and $c[q](x)$, respectively.

In this system, typing judgements are of the form $$\Theta; \Gamma \proves P$$, where $\Theta$ is a context of process variables and $\Gamma$ is a context of channels.
That is, the judgement is whether $P$ is a well-typed process.
We also elaborate the form $(\nu s) P$ into $(\nu s : \Gamma) P$, where $\Gamma$ can only contain channels of the form $s[p]$ for some set of roles $p$.

With this in mind, we can write the following intuitive endpoint projection of the example above:
$$
\cdot; \left\{
  \begin{array}{rcl}
    s[\textsf{Alice}] & : & \textsf{Bob}\oplus\langle \textsf{int}\rangle. \textsf{Cathy}\&(\textsf{int}). \textsf{end}\\
    s[\textsf{Bob}]   & : & \textsf{Alice}\&(\textsf{int}). \textsf{Cathy}\oplus\langle \textsf{int}\rangle. \textsf{end}\\
    s[\textsf{Cathy}] & : & \textsf{Bob}\&(\textsf{int}). \textsf{Alice}\oplus\langle \textsf{int}\rangle. \textsf{end}\\
  \end{array}\right\}
\proves
\left(
  \begin{array}{ll}
      & s[\textsf{Alice}][\textsf{Bob}]\langle n\rangle. s[\textsf{Alice}][\textsf{Bob}](z).0~\textcolor{gray}{\textit{-- Alice's Program}}\\
    | & s[\textsf{Bob}][\textsf{Alice}](x). s[\textsf{Bob}][\textsf{Cathy}]\langle \textsf{enc}(x)\rangle. 0~\textcolor{gray}{\textit{-- Bob's Program}}\\
    | & s[\textsf{Cathy}][\textsf{Bob}](y). s[\textsf{Cathy}][\textsf{Alice}]\langle \textsf{enc}(y)\rangle.0~\textcolor{gray}{\textit{-- Cathy's Program}}\\
  \end{array}
\right)
$$
This typechecks.
In fact, this has really good properties: the environment that we provide has the \textbf{terminating} property from~\citet{ScalasY19}, which proves that our projections are deadlock-free and always terminate.
Moreover, we should be able to show that this is true of any \textsf{let}-free choreography, and that every choreography typechecks under a context with the \textbf{deadlock-free} property.

\newcommand{\ET}{\mathcal{E}\mathcal{T}}
\newcommand{\EP}{\mathcal{E}\mathcal{P}}
\newcommand{\CT}{\mathcal{C}\mathcal{T}}
\newcommand{\CP}{\mathcal{C}\mathcal{P}}
To actually do the translation, I assume that expressions have two translations:
\begin{enumerate}
\item A type translation $\ET(s, p, \pi)$, where $s$ is a session name, $p$ is the principal who is running the expression, and $\pi$ is a proof that $\Gamma \proves e : \tau$ for some $\Gamma$, $e$, and $\tau$.
\item A program translation $\EP(s, s', p, q, \pi)$ where $s$ and $s'$ is a session name, $p$ is a principal who is running the expression, $q$ is a principal who should receive the output of the expression, and $\pi$ is a proof of $\Gamma \proves e : \tau$ for some $\Gamma$, $e$, and $\tau$.
\end{enumerate}
We furthermore assume that $\Gamma, \ET(s,p,\pi), s'[p] : q\langle \tau \rangle.T \proves \EP(s,s',p,q,\pi).P$ whenever $\Gamma, s'[p] : T \proves P$.

We then define two transltions
\begin{enumerate}
\item A type translation $\CT(s, p, e, \pi)$ where $s$ is a session name, $p$ is the principal being projected, $e$ is the environment who will receive the final value as a role, and $\pi$ is a proof that $\Delta; \Gamma \proves C : q.\tau$.
\item A program translation $\CP(s, p, e, \pi)$, where the variables are as above.
\end{enumerate}

The environment is a special principal.
It represents an outside actor, and so we don't expect it to do anything other than receive the computation.
(Perhaps it should have it's own session?)
Moreover, we only project a type for the environment, never a program.

We define them by induction on $\pi$.
All session variables bound by $\nu$ are assumed to be fresh.
\begin{mathpar}
  \CT\left(s, p, e, \infer{\infer{\pi}{\eppctxt{\Gamma}{p} \proves e : \tau}}{\Gamma;\Delta \proves p.e : p.\tau}\right) = e\langle\tau\rangle.0 \and
  \CP\left(s, p, e, \infer{\infer{\pi}{\eppctxt{\Gamma}{p} \proves e : \tau}}{\Gamma;\Delta \proves p.e : p.\tau}\right) = (\nu s' : \ET(s',p, \pi)) \EP(s', s, p, e, \pi).0 \\
  \CT\left(s, q, e, \infer{\infer{\pi}{\eppctxt{\Gamma}{p} \proves e : \tau}}{\Gamma;\Delta \proves p.e : p.\tau}\right) = 0 \and
  \CP\left(s, q, e, \infer{\infer{\pi}{\eppctxt{\Gamma}{p} \proves e : \tau}}{\Gamma;\Delta \proves p.e : p.\tau}\right) = 0\\
  \CT\left(s, e, e, \infer{\infer{\pi}{\eppctxt{\Gamma}{p} \proves e : \tau}}{\Gamma;\Delta \proves p.e : p.\tau}\right) = p(\tau).0\\
\end{mathpar}
\begin{mathpar}
  \CT\left(s, p, e, \infer{\infer{\pi_1}{\eppctxt{\Gamma}{p} \proves e : \tau}\\ \infer{\pi_2}{\Gamma, x : q.\tau; \Delta \proves C : r.\sigma}}{\Gamma;\Delta \proves p.e \to q.x; C : r.\sigma}\right) = q\langle\tau\rangle).\CT(s,p,e,\pi_2) \and
  \CP\left(s, p, e, \infer{\infer{\pi_1}{\eppctxt{\Gamma}{p} \proves e : \tau}\\ \infer{\pi_2}{\Gamma, x : q.\tau; \Delta \proves C : r.\sigma}}{\Gamma;\Delta \proves p.e \to q.x; C : r.\sigma}\right) = (\nu s' : \ET(s',p,\pi_1)) \EP(s', s, p, e, \pi_1). \CP(s, p, e, \pi_2)\\
  \CT\left(s, q, e, \infer{\infer{\pi_1}{\eppctxt{\Gamma}{p} \proves e : \tau}\\ \infer{\pi_2}{\Gamma, x : q.\tau; \Delta \proves C : r.\sigma}}{\Gamma;\Delta \proves p.e \to q.x; C : r.\sigma}\right) = p(\tau).\CT(s,q,e,\pi_2)\and
  \CP\left(s, q, e, \infer{\infer{\pi_1}{\eppctxt{\Gamma}{p} \proves e : \tau}\\ \infer{\pi_2}{\Gamma, x : q.\tau; \Delta \proves C : r.\sigma}}{\Gamma;\Delta \proves p.e \to q.x; C : r.\sigma}\right) = p(x). \CP(s, q, e, \pi_2)\\
  \CT\left(s, r, e, \infer{\infer{\pi_1}{\eppctxt{\Gamma}{p} \proves e : \tau}\\ \infer{\pi_2}{\Gamma, x : q.\tau; \Delta \proves C : r.\sigma}}{\Gamma;\Delta \proves p.e \to q.x; C : r.\sigma}\right) = \CT(s,r,e,\pi_2) \and
  \CP\left(s, r, e, \infer{\infer{\pi_1}{\eppctxt{\Gamma}{p} \proves e : \tau}\\ \infer{\pi_2}{\Gamma, x : q.\tau; \Delta \proves C : r.\sigma}}{\Gamma;\Delta \proves p.e \to q.x; C : r.\sigma}\right) = \CP(s, r, e, \pi_2)\\
\end{mathpar}

\begin{thm}[Simulation]
  \label{thm:simulation}
  If $C_1 \To C_2$ and $\infer{\pi}{\Gamma \proves C_1 : p.\tau}$, then $$\left(\mathop{|}_{q}\CP(s, q, e, \pi)\right) = \left(\mathop{|}_{q}\CP(s, q, e, \text{type-preserving}(\pi))\right)$$
  where $\text{type-preserving}(-)$ is the function on typing proofs coming from the constructive proof of Theorem~\ref{thm:equiv-pres-types}.
  Moreover, $\mathop{|}_{q}$ represents the parallel composition of each of the projections.
\end{thm}

\iffalse
\subsection{Refinement and Equivalence}

In previous work, the operational semantics of choreographies were predicated on an equivalence relation $\equiv$ which allowed for re-ordering programs due to concurrency.
This is similar to the equivalence relation that $\pi$-calculi are often defined up-to.
In these calculi, this is used to ensure that, for instance $P|Q$ behaves the same as $Q|P$.
However, this is mostly for convenience in writing the operational semantics, so that we don't have to write rules for both the cases where $P$ sends and $Q$ receives, and vice-versa.
Since we do not have such difficulties in writing the operational semantics of our choreography language, we do not include this notion of our equivalence in our semantics.

However, that does not mean that our choreographies should not respect that notion of equivalence.
Indeed, the equivalence of previous works (which we refer to as \emph{strong} equivalence) says that the two choreographies express \emph{the same} protocol.
Here is strong equivalence in our setting:
\noindent\begin{mathpar}
  \infer{ }{C \ceq_S C} \and \infer{C_1 \ceq_S C_2}{C_2 \ceq_S C_1} \and \infer{C_1 \ceq_S C_2 \\ C_2 \ceq_S C_3}{C_1 \ceq_S C_3} \\
  \infer{C_1 \ceq_S C_2}{\send{\own{p}{e}}{\own{q}{x}}; C_1 \ceq_S \send{\own{p}{e}}{\own{q}{x}}; C_2} \and
  \infer{C_{11} \ceq_S C_{21}\\ C_{12} \ceq_S C_{22}}{\cif{\own{p}{e_1}}{C_{11}}{C_{12}} \ceq_S \cif{\own{p}{e_1}}{C_{21}}{C_{22}}} \and
  \infer{C_{11} \ceq_S C_{21}\\ C_{12} \ceq_S C_{22}}{\clet{X}{C_{11}}{C_{12}} \ceq_S \clet{X}{C_{21}}{C_{22}}} \\
  \infer{p \neq r\\ q \neq r\\ p \neq s\\ q \neq s}{\send{\own{p}{e_1}}{\own{q}{x}}; \send{\own{r}{e_2}}{\own{s}{y}}; C \ceq_S \send{\own{r}{e_2}}{\own{s}{y}}; \send{\own{p}{e_1}}{\own{q}{x}}; C} \and
  \infer{p \neq q\\ p \neq r}{\cif{\own{p}{e_1}}{\send{\own{q}{e_2}}{\own{r}{x}}; C_1}{\send{\own{q}{e_2}}{\own{r}{x}}; C_2} \ceq_S \send{\own{q}{e_2}}{\own{r}{x}}; \cif{\own{p}{e_1}}{C_1}{C_2}} \and
  \infer{p \neq q }{\cif{\own{p}{e_1}}{(\cif{\own{q}{e_2}}{C_1}{C_2})}{\cif{(\own{q}{e_2}}{C_3}{C_4})} \ceq_S \cif{\own{q}{e_2}}{(\cif{\own{p}{e_1}}{C_1}{C_3})}{(\cif{\own{p}{e_2}}{C_2}{C_4})}} \and
  % \infer{ }{\cif{\own{p}{e}}{(\clet{X}{C_1}{C_2})}{(\clet{X}{C_1}{C_3})} \ceq_S \clet{X}{C_1}{(\cif{\own{p}{e}}{C_2}{C_3})}} \and
  % The above feels like it should be fine, but it's not in my current Coq code, so I'm commenting it out for now.
  \infer{p \notin \textsf{TN}(C_1)\\ q \notin \textsf{TN}(C_1)}{\clet{X}{C_1}{(\send{\own{p}{e}}{\own{q}{x}}; C_2)} \ceq_S \send{\own{p}{e}}{\own{q}{x}}; \clet{X}{C_1}{C_2}} \and
  % \infer{X \notin \text{FCV}(C_2)\\ Y \notin \text{FCV}(C_1)}{\clet{X}{C_1}{(\clet{Y}{C_2}{C_3})} \ceq_S \clet{Y}{C_2}{(\clet{X}{C_1}{C_3})}}
  %similar to the above
\end{mathpar}

The name ``strong equivalence'' suggests the existence of a counterpart \emph{weak} equivalence.
Indeed, there is such a notion.
It is intended to capture the idea that two choreographies express \emph{equivalent} protocols.
\akh{As far as I'm aware, the notion of equivalent protocols isn't something that's been explored in a lot of detail before.
  Then again, I haven't looked particularly hard.
  Perhaps something worth looking in to?}
Here is weak equivalence for this language:
\noindent\begin{mathpar}
  \infer{ }{C \ceq_W C} \and \infer{C_1 \ceq_W C_2}{C_2 \ceq_W C_1} \and \infer{C_1 \ceq_W C_2 \\ C_2 \ceq_W C_3}{C_1 \ceq_W C_3} \\
  \infer{C_1 \ceq_W C_2}{\send{\own{p}{e}}{\own{q}{x}}; C_1 \ceq_W \send{\own{p}{e}}{\own{q}{x}}; C_2} \and
  \infer{C_{11} \ceq_W C_{21}\\ C_{12} \ceq_W C_{22}}{\cif{\own{p}{e_1}}{C_{11}}{C_{12}} \ceq_W \cif{\own{p}{e_1}}{C_{21}}{C_{22}}} \and
  \infer{C_{11} \ceq_W C_{21}\\ C_{12} \ceq_W C_{22}}{\clet{X}{C_{11}}{C_{12}} \ceq_W \clet{X}{C_{21}}{C_{22}}} \\
  \infer{q \neq r\\ p \neq s\\ q \neq s}{\send{\own{p}{e_1}}{\own{q}{x}}; \send{\own{r}{e_2}}{\own{s}{y}}; C \ceq_W \send{\own{r}{e_2}}{\own{s}{y}}; \send{\own{p}{e_1}}{\own{q}{x}}; C} \and
  \infer{p \neq r}{\cif{\own{p}{e_1}}{\send{\own{q}{e_2}}{\own{r}{x}}; C_1}{\send{\own{q}{e_2}}{\own{r}{x}}; C_2} \ceq_W \send{\own{q}{e_2}}{\own{r}{x}}; \cif{\own{p}{e_1}}{C_1}{C_2}} \and
  \infer{p \neq q }{\cif{\own{p}{e_1}}{(\cif{\own{q}{e_2}}{C_1}{C_2})}{\cif{(\own{q}{e_2}}{C_3}{C_4})} \ceq_W \cif{\own{q}{e_2}}{(\cif{\own{p}{e_1}}{C_1}{C_3})}{(\cif{\own{p}{e_2}}{C_2}{C_4})}} \and
  % \infer{ }{\cif{\own{p}{e}}{(\clet{X}{C_1}{C_2})}{(\clet{X}{C_1}{C_3})} \ceq_W \clet{X}{C_1}{(\cif{\own{p}{e}}{C_2}{C_3})}} \and
  % The above feels like it should be fine, but it's not in my current Coq code, so I'm commenting it out for now.
  \infer{ q \notin \textsf{TN}(C_1)}{\clet{X}{C_1}{(\send{\own{p}{e}}{\own{q}{x}}; C_2)} \ceq_W \send{\own{p}{e}}{\own{q}{x}}; \clet{X}{C_1}{C_2}} \and
  % \infer{X \notin \text{FCV}(C_2)\\ Y \notin \text{FCV}(C_1)}{\clet{X}{C_1}{(\clet{Y}{C_2}{C_3})} \ceq_W \clet{Y}{C_2}{(\clet{X}{C_1}{C_3})}}
  % Similar to the above
\end{mathpar}
\noindent The main difference here is that, in the first rule, $p$ can now be equal to $r$, allowing us to swap two sends by the same principal.
This changes the protocol from that sender's point-of-view, but does not change anything from anyone else's point-of-view.
In several other rules, similar changes have been made.
It is clear that the following lemma holds:
\begin{lem}[Strong Equivalence Implies Weak Equivalence]
  $C_1 \ceq_S C_2$ implies $C_1 \ceq_W C_2$.
\end{lem}

Furthermore, we want our other constructions to respect these notions of equality.
We express that in the following theorems:
\begin{thm}
  If $\Gamma; \Delta \proves C_1$ and $C_1 \ceq_W C_2$ then $\Gamma; \Delta \proves C_2$.
\end{thm}
\begin{cor}
  If $\Gamma; \Delta \proves C_1$ and $C_1 \ceq_S C_2$ then $\Gamma; \Delta \proves C_2$.
\end{cor}

\begin{thm}[Church-Rosser-Like Diamond Property]
  Let $X \in \{W,S\}$.
  If $C_1 \ceq_X C_1'$ and $C_1 \To C_2$, then $C_1' \To^\ast C_2'$ where $C_2 \ceq_X C_2'$.
\end{thm}

\begin{thm}[Endpoint Projection of Strong Equivalence]
  If $C_1 \ceq_S C_2$ then for every thread $p$ $\transb{C_1}_p \equiv \transb{C_2}_p$ where $\equiv$ is the equality that the target process calculus is defined up-to.
\end{thm}
\begin{thm}[Endpoint Projection of Weak Equivalence]
  If $C_1 \ceq_W C_2$ then for every thread $p$ $\transb{C_1}_p \cong \transb{C_2}_p$ where $\cong$ is some equivalence on the target process calculus.
\end{thm}
\fi

\subsection{Example Programs}
A program that sends a variable around in a triangle.
$$\send{\own{\textsf{Alice}}{n}}{\own{\textsf{Bob}}{x}};~\send{\own{\textsf{Bob}}{\textsf{enc}(x)}}{\own{\textsf{Cathy}}{y}};~\send{\own{\textsf{Cathy}}{\textsf{enc}(y)}}{\own{\textsf{Alice}}{z}}; \own{\textsf{Alice}}{z}$$

A trusted-third-part program:
$$\send{\own{A}{3}}{\own{C}{x}};~\send{\own{B}{\lambda x.\,e}}{\own{C}{f}}; \own{C}{f~x}$$

\bibliography{bibliography/akh}

\end{document}


%%% Local Variables:
%%% mode: latex
%%% TeX-master: t
%%% eval: (push '("\\To" . ?⇒) prettify-symbols-alist)
%%% eval: (push '("\\cleq" . ?≼) prettify-symbols-alist)
%%% eval: (push '("\\ceq" . ?≈) prettify-symbols-alist)
%%% TeX-master: t
%%% End:

