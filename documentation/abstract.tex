Choreographies are a programming paradigm for writing concurrent and distributed programs with message passing.
Unlike process calculi, which are usually used to study concurrent programs, they allow a programmer to write a single program for the entire system.
This not only is a more convenient programming syntax, it also guarantees that programs are deadlock-free by construction.
However, previous studies of choreographic programming use imperative languages and only allow sending of atomic messages.
In this paper we study choreographic programming from a functional point of view.

We design a choreographic language in which messages are programs in some simply-typed language.
We can think of this as lifting a single-process functional language to a multi-process or distributed language, where each thread or node is programmed in the original language.
This lifted choreographic language is typed using the types of the message language augmented with \emph{modalities}.
This gives a syntactic, but informal, connection between multi-modal logics and message-passing concurrency.

Moreover, the theory in this document is formalized in Coq.
This gives us much more confidence in the theory presented here.
This also answers the call to action by \citet{Cruz-FilipeMP19}.
They note that there have been several high-profile cases of theorems about $\pi$-calcluli being disproven after publication, and suggest encoding choreographies in Coq to ensure that the same does not happen there.

This document in particular is designed for internal use: it records what we have done and communicates results with collaborators.
The writing will be rough, and the explanations sparse.

%%% Local Variables:
%%% mode: latex
%%% TeX-master: "PACMTRCameraReady"
%%% End:
